\let\textcircled=\pgftextcircled
\chapter{CERN and Project Planning}
\label{chap:ch3}

\section{Introduction and Overview}
The European Council for Nuclear Research is an International research organization that operates the largest particle physics laboratory in the world.  The main function of CERN is to create particle accelerators and other infrastructure needed for state-of-the-art fundamental research in high-energy physics and as a result numerous experiments have been constructed at CERN through international collaborations. Founded in 1954, the CERN laboratory sits astride the French-Swiss border near Geneva. It was one of the Europe's first joint ventures and now has 23 member states. At an intergovernmental meeting of UNESCO in Paris in December 1951, the first resolution concerning the establishment of a European Council for Nuclear Research was adopted.

Today, the understanding of matter at it's most fundamental level due to the efforts by CERN goes much deeper than the nucleus, and CERN's main area of research is particle physics. Physicists and engineers at CERN uses the world's largest and most complex scientific instruments to study the basic constituents of matter the fundamental particles. Subatomic particles are made to collide together at close to the speed of light. The process gives us clues about how the particles interact, and provides insights into the fundamental laws of nature.

The instruments used by CERN are purpose-built particle accelerators and detectors. The Accelerators boosts beams of particles to high energies before the beams are made to collide with each other or with stationary targets. the Detectors observe and record the results of these collisions.

The hierarchical structure of CERN is divided into sectors, which are further divided into departments, group's ad sections. The relevance for the thesis is attached to the Accelerators and Technology ATS, Sector which is responsible for the operation and exploitation of the whole accelerator complex, in particular the Large Hadron Collider (LHC) and for the development of new projects and technologies. The ATS sector comprises of the Beams, Engineering and Technology departments respectively. The focus for this thesis is on project management for nuclear infrastructure in the Engineering department, which provides CERN with the engineering competences, infrastructure systems and technical coordination required for design, installation, operation, maintenance and dismantling phases of the CERN accelerator complex and its experimental facilities. The activity of project management is associated with Administration, Resources and Performance (ARP) group, which is the backbone of the Engineering department, facilitating administration and planning, coordination of CERN's technical infrastructure and management of non-beam facility development projects and quality management support.


The projects and quality management support section has expertise in several domains of technical management, including training on project management, systems engineering, requirements engineering and quality management. Project leadership and support for multi-disciplinary non-beam facilities related project are also provided by the group.

Projects at CERN involves complex technical systems, while focusing on ORAMS/ RAMS, interoperability, reliability, availability, maintainability and safety in project management. CERN uses historical data, technical and professional expertise, MS-Project and standard classical tools such as WBS and Gantt chart for project planning and management. 

It has also developed a dedicated framework called OpenSE - an open, lean, and participative approach to systems engineering. The development of this dedicated framework is motivated by the observation that project management and systems engineering methodologies by PMI's Project Management Body of Knowledge or more specialized ones such as the NASA's Systems Engineering Handbook, which are not very suited to development of complex scientific facilities. It should be noted that these scientific facilities and/or systems are complex one-of-a-kind projects and includes several safety and other potential risks owing to complex scientific environment. 

\section{Background and Research}

Project planning at CERN \cite{article} involves using traditional and effective methodologies for project planning using guidelines of an OpenSE tool \cite{opense} created by CERN for project management of large and complex scientific projects.They define the planning framework which involves a detailed description of the macro and micro-activities in order to ensure effective completion of the project. It comprises of the three components: the people who will do the planning, a process which specifies what the project team needed to do to manage time and resources, and the software tools and hardware that helps the project team to implement the process.

The project management approach taken at CERN for project planning and control in which two types of project schedules are prepared firstly, the Master schedule, which is a master plan at a strategic level for the project and provides a project road-map with an intuitive approach. Secondly, the coordination schedule is created through a Gantt chart using Microsoft Project, displaying activity network at tactical level and more of an analytical approach.

In coordination Planning \& Scheduling is a three step process which is followed:
 
\begin{enumerate}
	\item Identifying the project activities using an analogical approach, Work breakdown Structure (WBS) based on PMBOK standards. It is a systematic approach based on the global lessons learned and collected by the Project Management Institute (PMI). Thereafter, the work packages are defined along with the activities and final deliverables of the project.
	
	\begin{figure}
		\centering
		\includegraphics[height=0.43\textheight]{fig03/fig14ch3.png}
		\mycaption[WBS]{WBS}
	\end{figure}

\item Identifying the resources available, estimating the resources required (based on historical data and technical and professional expertise) and these resources are assigned to the respective activities. 

\item  Scheduling the activities through the coordination schedule in a Gantt chart created in Microsoft software tool. It helps in estimating the duration of activities and also define the technical constraints between the activities. Furthermore, earliest/latest start/finish dates and the critical path are defined using Precedence Diagramming method and Gantt charts.

\end{enumerate}
\vspace{1cm}

The Project costing can be explained in a three step process - 

\begin{enumerate}
	\item Estimating the resources required to perform the project through various studies involved such as conceptual, feasibility and design etc. It answers to which costs to be taken into account and it is only  the chargeable cost that is considered. 
	
	\item Budgeting the resources allocated to the project, and the cost estimates becomes the budget document created in a MS-Excel Spreadsheet. The budget document is authored and verified by a project manager and a few key study members, then validation is done by a project team at a work package level and by the study manager and a project manager at the cost center level.
\end{enumerate}

The potential risks associated with the projects are related to quality, cost and schedule with an impact on use and coordination of activities and resources. The projects at CERN requires strong coordination among the technical groups and project management teams. To coordinate the project requires an overall view which can only be achieved by collecting and summarizing detailed information, which in turn is possible only if the historical data and information is coherent across the project.

The traditional project planning methodologies puts special emphasis on linear processes, upfront planning and prioritization. As per traditional classical methods like WBS, Gantt chart, time and budget are fixed, and requirements are variable due to which it often faces budget and timeline issues. The traditional classical methods are not able to interpret the uncertainty with the activities done in parallel by different cross functional teams in case of involving multiple stakeholders, hence any delay in work can lead to additional complexities and will have impact on both time and budget. All these risks cannot be reflected upon the Gantt chart which becomes difficult to monitor and control. 

	\begin{figure}
	\centering
	\includegraphics[height=0.13\textheight]{fig03/fig15ch3.png}
	\mycaption[Imprecision-Uncertainty]{Imprecision-Uncertainty}
\end{figure}


Use of advanced planning methodologies like Monte-Carlo can address these uncertainties that pose potential risks for the Nano-lab project. Monte-Carlo simulation model and a tool created in MS-excel are used for complex estimation of the scenario that involves a high degree of complexity and uncertainty to analyze the likelihood of meeting the objectives. The focus of the thesis is to use advanced project planning methodologies using a case study for a nuclear infrastructure project at CERN to mitigate shortfalls of traditional project planning methodologies heavily reliant on historical estimates and cannot take into consideration the uncertainty during the entire project life cycle. This main drawback is addressed by Monte-Carlo simulation model and an MS-Excel which is explained in subsequent chapters.








