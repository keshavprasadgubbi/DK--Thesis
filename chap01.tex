%
% File: chap01.tex
% Author: Victor F. Brena-Medina
% Description: Introduction chapter where the biology goes.
%
\let\textcircled=\pgftextcircled
\chapter{Fundamental Concepts of Project Management}
\label{chap:intro}

%\initial{S}ince this thesis study as well as the master's program is about Renewable energy, it's a good idea to start with the general context of renewable energy. %We will begin with my own quotation about renewable energy.\textcolor{brown}{'The greatest worry of man's imagination should be a world without energy sources. Those who think outside the box thought about the future with renewable energy sources'} %\textbf{(Kaaya Ismail)} 
%We will begin with a brief  description of the laboratory(IRDEP) were this work has been carried and  the funding laboratory(IPVF). We also present the objective and motivation of our study. 

 
 %==============================
%\begin{figure}
%    \centering
%  \includegraphics[height=0.43\textheight]{fig01/Clean-Energy}
%  \mycaption[Clean-energy]{From {http://www.vroc.ca/pir/en/clean-energy-canada-possible-near-future/}}
% \end{figure}
%================================
 \section{Project Management}
 %\section*{IRDEP}
 \label{sec:sec001}
 A project is a temporary endeavor undertaken to create a unique project, service, or result \cite{PMI-Website}.

According to ISO 21500, project consists of unique set of processes consisting of coordinated and controlled activities with start and end dates, performed to achieve project objectives. Achievement of the Project objectives requires the provision of deliverables conforming to specific requirements. A Project may be subjects to several constraints. Although many of the projects can be similar, or each project can be unique too. Project differences can occur due to the following reasons:
 \begin{itemize}
     \item Provision of the deliverables,
     \item Stakeholders influences, 
     \item Usage of Resources,
     \item Constraints,
     \item Tailoring of processes to ensure deliverables.
 \end{itemize}
     

Every project has a predetermined beginning and end dates and mostly project is divided into several phases along with defined objectives and goals of the project.

\say{Project management is the art of creating the illusion that any outcome is the result of a series of predetermined, deliberate acts when, in fact, it was damn luck.} - Harold Kerzner

Project management is the application of the various methods, techniques, tools and competencies to a project. It is a discipline that gives principles, techniques and tools to accomplish project goals within time and budget constraints. Project management involves integration of various phases of the complete project life cycle. Project management consists of number of interlinking processes. A process is a sequence of actions who work towards same goals and objectives. These processes should be aligned in a systemic view. Each phase of the project life cycle have specific deliverables and should be reviewed regularly during the project to meet all stakeholders requirements.

Project management comprises of planning, organizing, motivating and controlling all the resources to achieve intended goals. The first step of the project management is to ensure that the project delivers within the pre-defined limits and constraints. They could range from time to budget and other situational risks. The subsequent step is to optimize the allocation of resources and to integrate the inputs required to achieve those pre-defined goals / requirements. To accomplish these goals, project manager has to control and monitor them actively. 

Project management enables the managers to run a project from beginning to the end phase in an efficient and effective manner keeping in mind all stakeholders interests.

According to the Project Management Institute \cite{PMBOK2013}, the term Project Stakeholder refer to "an individual, group, or organization who may affect, be affected by, or perceive itself to be affected by a decision, activity, or outcome of a project". ISO 21500 uses a similar definition. 

Stakeholders for the projects are individual or group entities whose interest lies within a given project. They can be both internal and external to an organization which either funds a project, have interest or gain upon success of the project or may have a positive or negative influence on project completion,

Owing to the dynamic and complex nature of business needs, managing resources and ensuring effective communications among team is a real challenge. Hence, Project management has become eminent for managing performance of business and industrial needs.
As every project is unique, even the approach defined to project management can be different. It depends on the type of business, sector and constraints defined by the clients. 
 
Project management \cite{activecolab} is a combination of art and science, as so many authors has stated that it is a mix of knowledge and skills , some acquired , others developed , some intrinsic characteristics traits of any individual or a group of persons, tools and techniques which constitutes of scientific, qualitative and quantitative which supports the decision-making process in project management.

Often, project management is defined by a triangle called "Triple Constraint" \cite{Baratta}. The three most important factors are time, cost and scope. These three factors form the vertices of the triangle with quality as the central theme. 


\begin{figure}
\centering
  \includegraphics[height=0.23\textheight]{fig01/fig1ch1}
  \mycaption[The Triple Constraint]{The Triple Constraint}
 \end{figure}


The triple constraint has four elements:
\begin{enumerate}
    \item The project must complete within the allocated cost.
    \item Project must be completed on time. 
    \item Project deliverables must be within the project scope.
    \item Project must meet requirements of user quality.
\end{enumerate}
In recent years the project management triangle has given a way to the project management diamond \cite{Shenhar} with cost, time, scope and quality as the four vertices and user expectations as the central theme. No two users have same requirements and expectations, it is important have a clear understanding of user generic and specific requirements. 



 \begin{figure}
\centering
  \includegraphics[height=0.23\textheight]{fig01/fig2ch1}
  \mycaption[The Project Management Diamond]{The Project Management Diamond}
 \end{figure}

\subsection{Project Management Life-cycle}
\label{subsec:subsec01}
According to the PMBOK \cite{PMBOK2013}, the PMBOK guide (the standard guidebook for project managers around the world) the project management lifecycle comprises of the five types of processes such as Initiating, planning, executing, monitoring and controlling and closing. These processes are elaborated as below: 
 
It is important to define the goals, objectives and critical success factors of the project:
\begin{enumerate}
    \item \textbf{Initiating:} In the initiation phase, there is a need to identify the project goals, after carefully investigating all the options to come up with the meaningful solution. Recognize and define the beginning of the project and the smooth continuity of one phase to another during the project. Initiation process keeps the team focused on the project goals. Initiating is the most critical stage in project management. It also halts the project if it fails to meet the customer expectations. It will create a domino effect, disrupting all the following stages as well the final outcomes.

    \item \textbf{Planning:} After clearly defining the scope of the project, the next phase is planning. It is the second most important phase of every project lifecycle because each project is unique and requires special approach. It is an ongoing process which continues through the entire project lifecycle. It answers all the questions as we create a feasible scheme which includes clearly defined activities, cost estimates, schedule development and resource planning.

    \item \textbf{Executing:} It is the part of the project management lifecycle where we physically construct the deliverables and present them to the user, who then decides upon it. It is usually the longest phase and also directly depends upon the duration of the project. The project manager controls activities, resources and costs while the team is performing the required activities. To carry out the process and ensuring accurate and complete information flow and team formulation.

    \item \textbf{Monitoring and Controlling:} Despite good planning and careful execution, projects fails if they lack control, mechanisms for the processes involved within the project. Monitoring the key performance indicators for the project and control the quality of the project results, observing the significant changes and making necessary adjustments as per requirements. To keep everything under control following checks are made, collecting data from spreadsheets and completed tasks, comparing it with the plan (like schedule and budget estimates). Another technique called Earned Value Management is used to measure performance of the project. It is a systemic process used to find variances in projects based on differences between task performed and tasks planned. It is used to control the schedule and the costs and can be useful for project forecasting too.

    \item \textbf{Closing:} Project closure allows the team to evaluate and document the complete information about the project. In the closing phase, we assess the project performance (in terms of objectives, scope, deliverables, schedule and cist), evaluate team member’s performance, list down project achievements and failures and keep data for future prospective projects. Formal acceptance of the project deliverables and dissolution of all the elements required to run the project. To gather all necessary data to validate the completed project. 
    
    \begin{figure}
    	\centering
    	\includegraphics[height=0.65\textheight]{fig01/fig3ch1}
    	\mycaption[Project Life Cycle]{Project Life Cycle}
    \end{figure}
    

\end{enumerate}


Project life cycle is the series of the phases that the project which passes from its start to its completion. It also provides the basic framework for project management. The phases of the project lifecycle may be sequential, iterative, or overlapping.

Project life cycles can be predictive or iterative, within a project life cycle, there are generally one or more phases associated with the development of the results. These are called development life cycle, they can be predictive, iterative, incremental, adaptive, or hybrid model.

\begin{itemize}
    \item Predictive life cycle also called Waterfall, the project scope, schedule and costs are determined in the early phases of the project.

    \item Iterative life cycle, project scope is generally determined in early phases of the project lifecycle, but schedule and cost are modified in a routine manner.

    \item Incremental life cycle \cite{PMBOK2013}, deliverable is produced through a series of iterations that successively add functionality within a predetermined time frame and is considered complete only after the final iteration takes place.

    \item Adaptive lifecycle are agile, iterative or incremental. The project scope is defined in a detail manner and agreed upon the start of an iteration. Adaptive lifecycles are also referred to as agile.
    \item Hybrid lifecycle, is a combination of a predictive and an adaptive life cycle. 
    
    \begin{figure}
    	\centering
    	\includegraphics[height=0.65\textheight, angle = 90]{fig01/fig4ch1}
    	\mycaption[How Processes Overlap Each Other]{How Processes Overlap Each Other}
    \end{figure}
  
\end{itemize}

The process does not necessarily follow the chronological pattern. As in due course of a project, sometimes things do get out of hands and are difficult to control. That is why processes usually overlaps throughout different phases of the project and ultimately become interdependent. All phases of the project life cycle interacts and are linked by their results. Subsequently the project manager sometimes has to return to earlier phases, makes necessary adjustments and amendments and keep continuing with the project management processes. 

 
For example, during the controlling and monitoring phase, if we need to allocate more resources to some activity, we need to modify the planning phase and amend resources planned for that activity. Else, inconsistency can have a major effect on entire project lifecycle.

 \begin{figure}
 	\centering
 	\includegraphics[height=0.45\textheight]{fig01/fig5ch1}
 	\mycaption[Flow of Project Processes]{Flow of Project Processes}
 \end{figure}
 Processes are linked to each other by their outcomes, as output of one process becomes the input for another. For example, the planning process provides the execution process with a pre-determined project plan. But planning process plays an important role throughout as it updates the plan as the project makes progress into further processes. Project planning is the most significant process as it lays down the project plan with detail of all the work packages and predicts possible obstacles need to overcome along the project run.

It is not easy to achieve success with the project, if we rely completely on clearly defined processes. Project management is both a science and an art.

\subsection{Role of a project manager}
\label{subsec:subsec02}

The role of the project manager involves planning and organizing the resources and schedule necessary for accomplishment of projects.

As per Project Management Institute, a project manager \cite{pmanager} is organized, passionate and goal-oriented who understands what projects have in common, and their strategic role in how organizations succeed, learn and change.

They are change agents, they commit to the project goals, and use their skills and expertise to promote a sense of share purpose for the entire project team.
They work well under pressures and easily adapt with change and complexity in dynamic environments. They have good people skills and they possess abroad tools and techniques to resolve complex interdependent activities into tasks and sub-tasks that are documented, monitored and controlled.

In late 1980s, Microsoft encountered a problem of coordination among different teams. They came up with a solution to appoint an individual who has significant authority as a leader and a coordination for the project named Excel. They horizontally deployed project manager, processes ran more smoothly and team satisfaction with work dynamics. Later, Microsoft adopted Project Manager as a new role.

The Project manager has several key roles and responsibilities:

\begin{itemize}
    \item \textbf{Activities and Resource Planning:}  Planning is instrumental for completion of a project. The prime task of a project manager is to define the project's scope and determine the availability of resources. They create a clear and concise document for guiding throughout the project execution and project control.

\item \textbf{Organizing and motivating a project team:} They are in charge of developing a plan to support the team in order to achieve the project goals and control the performance.

\item \textbf{Time Management:} Users evaluate the performance of the project depending on whether it has been delivered on time and on-time completion of the project is the pre-requisite for the success of the project.

\item \textbf{Cost estimates and budget allocation:}  A project manager ensures the completion of the project within defined and allocated budget. They frequently review the budget plans and forecasts to avoid cost overruns.

\item \textbf{Ensure user satisfaction:} Users satisfaction is the most important key performance indicator to evaluate the project success. A project manager avoids uncertainty and keep user informed at every phase of the project.

\item \textbf{Analyzing and managing project risks:} A project manager identifies and evaluates the potential risks and develop appropriate strategies to avoid and minimize the risks and their impact on the project. 

\item \textbf{Monitor progress:} A project manager need to control and analyze the key performance indicators related to the cost and time; take appropriate preventive and corrective measures. 

\item \textbf{Creating and managing reports and necessary documentation:} Creating and managing reports and necessary documentation: Finally, an experienced project manager provides relevant documentation with final reports and also identifies areas of development for future projects.
\end{itemize}

Project managers are integral part of all organization ranging from big ones to small ones, who are in-charge of complex and ambitious projects.

 \section{Planning \& Scheduling in Project Management}
 %\section*{IRDEP}
 \label{sec:sec002}
 
 
 Planning and scheduling are two distinct but inseparable aspect of Project management \cite{Moylan}. The process planning primarily medals with appropriate tools and techniques and methods to achieve the goals of the project. On the other hand, Scheduling deals with project baseline plan (Scope, cost and time) into an operational timetable. By merging together the project plan and the budget, schedule is the most important tool for project management. Further, integrated cost-time schedule serves as a fundamental basis for monitoring and controlling the project tasks thought out the project life cycle.
 
 \subsection{Relevance}
 
 The Planning process is the one of the most important phase of the Project management. To develop project management plan is the process of defining, preparing, and coordinating all other planning activities.
 
 The major benefit of the planning process is the creation of a comprehensive document which defines the entire project work and the how to perform the defined activities.
 
 \begin{itemize}
 	\item Planning process is performed once in the beginning of the project or at predefined points of the project.
 	\item  Project management plan can be both as a summary version or a detailed one.
 	\item Project management plan should have  a baseline (Scope, Time and Cost)
 	\item Project management plan can be updated as many times as deemed necessary. But, once it is baselines it can only be changed through internal control process.
 \end{itemize}
 
 \textbf{Project Schedule as a Model of Control:} The project planning \cite{Zwikael} process leads to creation of a schedule as a model for project control which has its occurrences in certain situations of project initiation and implementation, for tracking progress and monitoring performance. It is very crucial for adapting to changes to avoid any project delays and clear communication amongst the project teams and an important process for project evaluation.
 

 \subsection{Criticality}
 
 Project planning \cite{resourceconstraint} process specifies the process to follow in future and assist with decision making in order to execute the entire project. The project manager are responsible for completion of the projects keeping in mind the interest of the stakeholders. They have to make sure that the plan is reliable and properly represents user's requirements. It is the most important and hence the critical process of the project management life cycle.
 

\section{General Project Planning Methodologies}
 %\section*{IRDEP}
 \label{sec:sec003}
 
 As project management began to be studied as a science and discovering various project archetypes, big corporations comes with different methodology and tools. As it spread over vast variety of industries, project managers tailored every approach and, methodology based on the industry requirements and different types of the projects.

For example, agile methodologies are effective on software development but it doesn’t work with infrastructure projects and vice-versa. Hence methods and techniques are bundled into methodologies to run a project consistently within time and within budget.

\emph{How to choose a right project management methodology?}
There are quite some good project planning methodologies available, so it can be difficult to choose the right one. \textbf{The focus of the thesis is primarily on project planning methodologies.}

A brief overview on different types of general project planning methodologies:

\subsection{Traditional Project Management Planning}

Traditional Project management is a universal practice which includes several techniques for planning, estimating and controlling activities. The aim of these techniques is to reach the desired project goals on time, within budget and in accordance to technical and managerial specifications. The ultimate goal is to make sure all the activities are carried out in a sequence.

In traditional planning, it is a sequence of steps taken to build the roadmap of the project and is one of the most fundamental step defining the performance of the project goals.

\begin{itemize}
    \item To identify the project activities. 
    \item To estimate the project duration.
    \item To determine the required resources. 
    \item To develop a sequence of project activities network. 
    \item To develop a documented project plan.
\end{itemize}

Traditional project management planning is the most common way to plan projects. It is not a methodology rather it is a collection of techniques, like WBS (Work Breakdown Structure), dividing into phases. The different tools used in traditional planning are:
\begin{enumerate}
    \item WBS - Work Breakdown Structure
    \item Project schedule and Plan - Gantt Chart 
\end{enumerate}

\subsubsection{ \textbf{Work Breakdown Structure (WBS):} }
 According to PMBOK \cite{PMBOK2013}, WBS is a deliverable-oriented hierarchical decomposition of work to be executed by the project team to accomplish the project objectives and create the required deliverables.

The purpose of WBS \cite{WBSs} is to define the project needs, to accomplish, and organize them into multiple levels, and displayed graphically in form of a chart.

 \begin{figure}
	\centering
	\includegraphics[height=0.55\textheight]{fig01/fig6ch1}
	\mycaption[WBS - Work Breakdown Structure]{Work Breakdown Structure}
\end{figure}
 
The key terminologies used with WBS are work, deliverable and work packages. In the context to the project management, these terms have specific definitions:
\begin{itemize}
    \item \textbf{Work:} According to PMBOK, work refers to "work products or deliverables that are the result of effort and not the effort itself". The work defines the end result of the activity. The work remains constant even though the amount of effort needed is variable.
    \item \textbf{Deliverable:} PMBOK \cite{projectvar} says that deliverable is "any unique and verifiable product, result, or capability to perform service that is required to be produced to complete a process, phase, or project". Deliverables may vary within the project.
    \item \textbf{Work Package:} According to PERT, they developed WBS, a work package is  "the work required to complete a specific job or process, such as a report, design , a documentation requirement or portion thereof, a piece of hardware , or a service". PMBOK provided a simpler definition - "a work package is the deliverable at the lowest level of the WBS."
\end{itemize}

\textbf{Characteristics of a WBS:}

\begin{itemize}
    \item \textbf{Hierarchy:} The WBS is a hierarchical nature divided into macro activity and micro activities and former is the sum of all the latter activities.
100\%: Every level of decomposition should make up for 100\% of the macro activity.
    \item \textbf{Mutually exclusive:} All elements of the WBS must be mutually exclusive. There must be no overlap in the deliverables as it avoid duplication of activities.
    \item \textbf{Outcome-focused:}  The WBS focus on the deliverables not on the activities necessary to reach the goal.
\end{itemize}

Work Breakdown Structure data is the input to create a project schedule.

\subsubsection{Gantt chart} 

Gantt chart is the most important technique in the traditional project management. It was created by Henry Gantt, which is who is considered as father of traditional project management.

It is an important tool for project planning, most useful ways of presenting tasks and activities of the project on a timeline. Currently, Gantt charts in excel. The Gantt chart is represented as -
 
\begin{figure}
	\centering
	\includegraphics[height=0.55\textheight]{fig01/fig7ch1}
	\mycaption[Gantt chart (GC)]{Gantt Chart}
\end{figure}
 
 
By looking at the Gantt chart, it displays - 

\begin{itemize}
    \item What are the project tasks are 
    \item Who is working on each task 
    \item How long each task could take 
    \item How tasks overlap and link with each other 
    \item The start and finish the date of the project
\end{itemize}

Gantt chart is used to track the projects schedules and make project management less cumbersome. It helps to understand the relationship between tasks in a clear manner, keep all the team members well informed and successfully complete the project.

In today's scenario due to COVID - 19, Even if we are using traditional tools like Gantt chart to manage the projects, as long as project team embraces changes, provides frequent and valuable output to the users, reflecting and adjusting as needed, it is compatible to agile methodology to manage projects. 

Traditional project planning tools in 21st century, as computer and internet became essentials parts of the business, processes became more complex and demanding, it no longer offers the best solutions for user's problems. The concept of traditional project planning tools has evolved and extended through different project management planning methodologies and frameworks.

Nevertheless, traditional management is still considered as foundation of all modern approaches and is still an inevitable methodology for big or complex infrastructure projects.
\\

\begin{tabular}{ |c|c|c| } 
	\hline
	\multicolumn{2}{|c|}{ Gantt Chart: An Overview} \\
	\hline
	Advantages & Disadvantages \\
	\hline
	Easy to organize ideas & Increasingly complex with demanding projects\\
	\hline
	Clear layout of the activities &  doesn't indicate actual quantity of work and resources \\
	\hline
	Helps to create realistic time schedules &constant updating due to changes in project planning \\
	\hline
	Highly visible  & Difficult to see all activities and details in one chart  \\
	\hline
\end{tabular}




\subsection{Deterministic Methodology - Critical Path method (CPM)}

Critical Path Method (CPM) is the most important deterministic methodology for project planning and scheduling. It is a mathematical algorithm that helps to analyze, plan and schedule complex projects.

Each project consists of tasks and activities that are interconnected and essential for project's success. CPM comes into play when the project gets complex and demanding when traditional planning tools becomes ineffective. At its core, CPM is a powerful tool that allows to identify the longest path of the planned activities critical to meet the deadlines and also identify the early start and finish dates. By determining the critical path, we will know the activities are critical in completing the project, and which activities do not pose any serious impact on the project development.

\textbf{How was CPM created? }

In late 50's, El Du Pont de Nemours, an American chemical company, were seriously behind schedule, and they needed help with it to come back on the timeframe to complete the project. They came up with a solution to divide their project into thousands of tasks, measure the time to complete each activity, to assess which activities are critical to the entire project. They called this technique Critical Path Method or CPM.It was first tested for as chemical plant construction project, and since then it is used as one of the most frequently used techniques of the project scheduling and planning. 

\subsubsection{CPM Approach\cite{CPM}} 

\begin{itemize}
	\item CPM calculates :
	\begin{itemize}
		\item The longest path of the planned activities to the end of the project.
		\item The earliest and latest that each activity can begin and end without making the project longer.
	\end{itemize}
		\item Determine the critical activities "the longest path".
		\item Prioritize the activities for the effective management and to shorten the planned critical of the project by -
		\begin{itemize}
			\item Pruning the critical activities 
				\item Fast tracking, i.e, performing more activities in parallel 
			\item Crashing the critical path by shortening the durations of the critical path activities by adding resources. 
		\end{itemize}
\end{itemize}



\subsubsection{Steps in Critical Path Method }

\begin{figure}
	\centering
	\includegraphics[height=0.55\textheight]{fig01/fig8ch1}
	\mycaption[CPM Workflow]{CPM Workflow}
\end{figure}



A critical path method include the following methods:

\begin{enumerate}
	\item \textbf{Identify activities:} By using the WBS, utilize the list if activities and identify them by name and a code, all activities must have a defined duration and end date.
		\item \textbf{Determine the sequence of the activities:} It is the most important step as it provides a clear view of the relationship between the activities and helps to establish the dependencies as some activities occurrence will depend on the completion of others.
			\item \textbf{Creating a network of activities:} After determining the interdependency amongst the activities, a network diagram is created, a critical path analysis chart; it allows you to use arrows to connect to the activities based on the dependency.
			
				\item \textbf{Determine the completion time for each activity:}  By estimating the time taken to complete each activity will help to determine the total time needed to complete the project.
					\item \textbf{Finding the critical path:} A network of activities will help to create the longest sequence "the critical path" using these parameters 
					
\end{enumerate}

\textbf{ES Early Start:} the earliest time to start a certain activity provided that the preceding one is completed \\
\textbf{EF Early Finish:} the earliest time required to the finish the activity  \\
\textbf{LF Late Finish :} the latest time required to finish the project without delays \\
\textbf{LS Late Start:} the latest start date when project can start without project delays. \\


\begin{figure}
	\centering
	\includegraphics[height=0.39\textheight]{fig01/fig9ch1}
	\mycaption[CPM Diagram]{CPM Diagram}
\end{figure}


If there is a delay in any task on the critical path, the entire project will be delayed. The critical path is the path where can be no delays. Not all of the project activities are equally important, while some of them have a huge impact on the critical path. The critical path method helps to determine which activities are “critical and which have “total float”. However, if any of the floating activities get seriously delayed, they become critical, and hence delays the entire project.

\textbf{How resource limitations effect the CPM?}

There are always certain limitations that affect our projects and create new dependencies. For example, resource limitations. In such scenario, the critical path changes into "critical resource path" where resources related to each activity become important part of the process.

As some of the tasks will have to be performed in a different order, which may cause delays and consequently, make the projects becomes longer than estimated.


Benefits of CPM are:
Although new techniques have been there in fast-paced technological advances, CPM still offers a number of advantages.
Prioritized tasks 
\begin{enumerate}
	\item Clarity on project's timeline for modifications 
	\item Risk assessment becomes easy 
	\item Redistribution of resources is efficient 
	\item Helps project panning team to stay focused

\end{enumerate}

Hence, CPM allows tom reschedule less important tasks and focus all efforts on critical path to optimize work thereby avoiding delays.

\subsection{Probabilistic   Methodology - Program Evaluation and Review Technique (PERT)}

The Program Evaluation and Review Technique PERT\cite{activecolab}, is a probabilistic methodology in project planning that helps to analyze, represent, and evaluate and estimate the time required to complete the project within defined time duration. PERT allows project planners to identify start and end dates, and ultimately contributes to reducing time and cost required to complete the project.

PERT was developed in 1958 by the US Navy as a part of the Polaris project. Their aim was to manage the Polaris submarine missile program. CPM was also developed around the same time.

Unlike CPM, which determines the longest path needed to finish the project, PERT provides three different time estimates. While PERT deals with unpredictable activities, but CPM deals with predictable activities. Hence, PERT deals with non-repetitive activities and CPM involves activities of repetitive nature. Owing to these differences, PERT is widely used in research and development projects and CPM heavily used in construction projects.

Steps in PERT:

\begin{enumerate}
	\item Identify specific activities and milestones: By listing out all the activities within a table, to get a clear overview of all the steps which can subsequently expand by adding information on sequence and time necessary to complete each activity.
	
	\item Determining the sequence of the activities: It is easy to predict order for some activities while others might need in-depth analysis to determine their order.
	\item Develop a network diagram: Once the sequence of the activities is defined, activities can be represented both series and parallel activities in the diagram.
	\item Each activity should represent a node in the network, and arrows are used to show relationships between the activities.
	\item Estimating the time required for each activity: What distinguishes PERT from the other techniques of project planning is its ability to deal with uncertainty in activity completion time. There are 3 time estimates PERT uses for each activity:
	\begin{itemize}
		\item \textbf{Optimistic time:} the shortest time in which activity can complete 
		\item \textbf{Most likely time:}the completion time that has highest probability 
		\item \textbf{Pessimistic  time:} the longest time in which the activity can complete 
	\end{itemize}
Once these 3 time estimates are identified, expected time is calculated for each activity by using the following weighted average as:\\
\\
$Expected time = \dfrac{(Optimistic + (4* Most likely) + Pessimistic)}{6} $
		\item \textbf{Identifying the critical path:}  By adding the times for the activities and determining the longest path, we can calculate the critical path. The critical path involves the total amount of time required for the project completion. The total project time is independent of the change in activities outside critical path.
		
		
		\begin{figure}
			\centering
			\includegraphics[height=0.23\textheight]{fig01/fig10ch1}
			\mycaption[PERT Diagram]{PERT Diagram}
		\end{figure}
		
\end{enumerate}


\subsubsection{Advantages}

At its core, PERT helps to control complex and ambitious projects whose objectives are highly critical in nature. It helps to determine the fastest route possible to complete the projects. In-depth analysis of the project by viewing the activities both independently and in connection with each other. This provides clarity on time and budget required to complete the entire project.
What-if analysis helps to identify all the possibilities and uncertainties related to the project. By trying different combinations and choosing the most useful possibility, we can eliminate risks. It also highlights the activities that require careful monitoring.


\subsubsection{Disadvantages}

Even though PERT has proven to be an effective methodology for reducing the expected project completion time, there are still some limitations associated with it.
Although PERT clearly defines all the activities on a project, it is sometime impossible to predict every step. As changes takes place within the project, it can seriously affect the initial PERT. While it is possible to make modifications but it leads to opportunity cost in terms of time.
Project planning managers make time estimates and as they heavily depend on the estimates based on the experience of the project managers.

Overall, PERT allows to have an idea of possible time variation and helps to assess the importance of the uncertainties along the entire project duration. Unlike most methodologies, PERT provides the flexibility to identify the best-case and the worst-case scenarios and to develop a strategy on how to best coordinate large-scale projects. 


\subsection{Computational algorithm - Monte Carlo Simulations}

The Monte Carlo Simulations is a computational algorithm used for simulations of various mathematical, statistical and scientific situation's during the project and is explained in several standards and PMI book\cite{Sandra}.

The technique (named after the Monaco town famous for its casinos) was initially developed by scientists working on the atom bomb during the second world war to game for possible scenarios in the outcome of the Manhattan project and the potential roadblocks that could arise during the execution of the secretive and high-stakes project. Since then, it is being utilized in vast number of fields. It utilizes the quantitative data and help in communicating and justifying the project variables in an explicit manner in comparison to forecast made on historical estimates.

It can be simply defined as statistical analysis optimized for approximation of quantitative problems on the basis of random sampling and probability. Based on the project completion variables, such as time and cost, they are represented through probability distributions. Monte-Carlo method can simulate the complete project for thousands of times depending on the nature of the project by selecting the random values each time from its probability distributions. After that, the process is repeated in order to calculate the overall completion variables of the project.

It will be discussed in great detail in subsequent chapters and its usage as an advanced planning methodology in project management for nuclear infrastructure at CERN.

