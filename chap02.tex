\let\textcircled=\pgftextcircled
\chapter{Monte Carlo Simulation}
\label{chap:litrev}
\section{Introduction}
Monte Carlo simulation is a computational technique for modeling and analyzing real-world systems and situations through projections of existing data for managing uncertainties during project management. The results from these simulations allow project-managers to identify, analyze, and assess possible risks and develop risk mitigation and contingency plans during the execution of complex projects. We can therefore account for risk in quantitative analysis and decision making to handle difficult situations and by incorporating such interventions, we can equip and empower the project-manager to lead the project towards successful completion. 

\begin{figure}
	\centering
	\includegraphics[height=0.23\textheight]{fig02/risk.png}
	\mycaption[MC Simulation Workflow]{MC Simulation Workflow \cite{MCRisk}}
\end{figure}

Monte Carlo simulation \cite{mcoutcomes} provides the decision-maker with a plethora of possible outcomes and their respective probabilities of occurrence for any given choice of action. These possibilities includes extremes - the outcomes of going for broke and the conservative option - along with all possible consequences for intermediary decisions. In the field of project management, Monte Carlo simulation can quantify the effects of risk and uncertainty in project schedules and budgets, giving the project manager a statistical indicator of project performance such as target completion date and budget.


\begin{figure}
	\centering
	\includegraphics[height=0.23\textheight]{fig02/monte-carlo-simulation.png}
	\mycaption[Range of Outcomes]{Range of Outcomes \cite{mcoutcomes}}
\end{figure}

Monte Carlo simulations are employed for effective risk management \cite{MCinvest} in wide-ranging fields such as manufacturing, engineering, finance, project management, energy, research and development, insurance, oil \& gas, and transportation. 

In this thesis, we have adapted the Monte Carlo simulation to be utilized as a forecasting tool for project scheduling and cost estimation, which is enunciated in detail in subsequent chapters. We have addressed uncertainties that poses potential risks to the project and quantifying useful likelihoods for meeting project milestones and immediate time and cost goals. It can also predict occurrence of scheduling and cost overruns. Hence, it enables to restructure the project as per updated requirements. Monte Carlo tools can also be further used for sensitivity and uncertainty analysis for risk variables of the project. 

\section{How MC Simulation Works}

Monte Carlo simulation helps to build models using probability distributions - for any factors that has inherent uncertainty. It then calculates results over many iterations, each time using a different set of random values from the probability functions. 

Based on the number of uncertainties and their ranges, a Monte Carlo simulation could involve thousands of iterations that produces distributions of possible outcome values. By using probability distributions, variables can have probabilities for occurrence of different outcomes. Probability distributions are a much more realistic way of describing uncertainty in variables in terms of time and cost.

The standard deviation of that probability quantifies the likelihood that the actual outcome being estimated will be something other than the mean or most probable event. Assuming a probability distribution is normally distributed, approximately 68\% of the values will lie within one standard deviation of the mean, about 95\% of the values will lie within two standard deviations, and about 99.7\% will fall within three standard deviations of the mean. This is known as the "68-95-99.7 rule" \cite{MCSimulation} or the "empirical rule."

The most common probability distributions \cite{MCSimulation} and certain important features related to them is described below.
\begin{enumerate}
	\item \textbf{Normal/Bell Curve:} The user simply defines the mean or expected value and a standard deviation to describe the variation about the mean. It is a symmetric curve and values at the center of distribution (near mean value) are ones most likely to occur in this kind of distribution. 
	
	\item \textbf{Lognormal Curve:} Unlike the normal distribution, it is an asymmetric curve since the values in this distribution are positively skewed. It is used to represent values that don't go below zero but have unlimited positive potential.
		
	\item \textbf{Uniform Curve:} All values have an equal chance of occurring, and the user simply defines the minimum and maximum. 
		
	\item \textbf{Triangular Curve:} The user defines the minimum, most likely, and maximum values. Values in the neighborhood of the most likely are more probable to occur. 
	
	\item \textbf{PERT Curve:}	The user defines the minimum, most likely, and maximum values, just like the triangular distribution. Values around the most likely are more likely to occur. However values between the most likely and extremes are more likely to occur than the triangular; that is, the extremes are not as emphasized. 
	
		\item \textbf{Discrete Curve:} The user defines specific values that may occur and the likelihood of each
\end{enumerate}

\begin{figure}
	\centering
	\includegraphics[height=0.23\textheight]{fig02/mc_distributions.jpg}
	\mycaption[Distributions]{Distributions \cite{MCSimulation}}
\end{figure}


\section{Advantages and Limitations of MC Simulation}
Monte Carlo simulation provides a number of advantages \cite{MCSimulation} over deterministic, or "single-point estimate" analysis like:
\begin{enumerate}
	\item \textbf{Probabilistic Results:} The result indicates not just what could happen, but also how likely each outcome is.
	 \item \textbf{Graphical Results:} Monte Carlo simulation generates graphs of different outcomes and their chances of occurrence. This is important for communicating findings to other stakeholders.
	 \item \textbf{Scenario Analysis:} In deterministic models, it's very difficult to model different combinations of values for various inputs to see the effects of truly different scenarios. Using Monte Carlo simulation, an analyst can quantitatively predict the inputs that had clustered values when certain outcomes occurred. This will be immensely valuable for pursuing further analysis.
     \item \textbf{Sensitivity Analysis:} Deterministic analyses makes it difficult to predict the variables that have the maximum impact on the outcome. In Monte Carlo simulation, it’s easy to see which inputs had the biggest effect on bottom-line results.
	 \item \textbf{Correlation of Inputs:} In Monte Carlo simulation, it's possible to model interdependent relationships between input variables. It's important for accuracy to represent how, in reality, when some factors goes up, others go up or down accordingly.
	 \item \textbf{Quantifiable Reasoning:} Monte Carlo simulation allows project-managers to better justify and communicate their arguments in the face of unrealistic project expectations.
\end{enumerate}

 

On the other hand, this approach also involves certain limitations like:
\begin{enumerate}
	\item The results are highly data-driven and data-dependent; therefore the quality of final estimates is directly impacted by quality and nature of input data.
	\item The Monte Carlo Simulation shows the probability of completing the tasks, not the actual time to complete.
	\item  This technique will not applicable for a single activity along with completion of risk assessments completed as an pre-existing requirement.

\end{enumerate} 



















