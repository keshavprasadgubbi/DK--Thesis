\title{MSc2 thesis}
%\title{University of Bristol Thesis Template}
\RequirePackage[l2tabu]{nag}% Warns for incorrect (obsolete) LaTeX usage
%
% File: memoirthesis.tex
% Author: Victor Brena
% Description: Contains the thesis template using memoir class,
% which is mainly based on book class but permits better control of 
% chapter styles for example. This template is an adaptation and 
% modification of Oscar's.
% 
% Memoir is a flexible class for typesetting poetry, fiction, 
% non-fiction and mathematical works as books, reports, articles or
% manuscripts. CTAN repository is found at:
% http://www.ctan.org/tex-archive/macros/latex/contrib/memoir/
%
%
% UoB guidelines for thesis presentation were found at:
% http://www.bris.ac.uk/esu/pg/pgrcop11-12topic.pdf#page=49
%
% UoB guidelines:
%
% The dissertation must be printed on A4 white paper. Paper up to A3 may be used
% for maps, plans, diagrams and illustrative material. Pages (apart from the
% preliminary pages) should normally be double-sided.
%
% Memoir class loads useful packages by default (see manual).
\documentclass[a4paper,11pt,leqno,openbib]{memoir} %add 'draft' to turn draft option on (see below)
%
%
% Adding metadata:
\usepackage{datetime}
\usepackage{ifpdf}
\ifpdf
\pdfinfo{
   /Author (Author's name)
   /Title (PhD Thesis)
   /Keywords (One; Two;Three)
   /CreationDate (D:\pdfdate)
}
\fi
%\mathsf{}
% When draft option is on. 
\ifdraftdoc 
	\usepackage{draftwatermark}				%Sets watermarks up.
	\SetWatermarkScale{0.3}
	\SetWatermarkText{\bf Draft: \today}
\fi
%
% Declare figure/table as a subfloat.
\newsubfloat{figure}
\newsubfloat{table}
% Better page layout for A4 paper, see memoir manual.
\settrimmedsize{297mm}{210mm}{*}
\setlength{\trimtop}{0pt} 
\setlength{\trimedge}{\stockwidth} 
\addtolength{\trimedge}{-\paperwidth} 
\settypeblocksize{634pt}{448.13pt}{*} 
\setulmargins{4cm}{*}{*} 
\setlrmargins{*}{*}{1.5} 
\setmarginnotes{17pt}{51pt}{\onelineskip} 
\setheadfoot{\onelineskip}{2\onelineskip} 
\setheaderspaces{*}{2\onelineskip}{*} 
\checkandfixthelayout
%
\frenchspacing
% Font with math support: New Century Schoolbook
\usepackage{fouriernc}
\usepackage[T1]{fontenc}
%
% UoB guidelines:
%
% Text should be in double or 1.5 line spacing, and font size should be
% chosen to ensure clarity and legibility for the main text and for any
% quotations and footnotes. Margins should allow for eventual hard binding.
%
% Note: This is automatically set by memoir class. Nevertheless \OnehalfSpacing 
% enables double spacing but leaves single spaced for captions for instance. 
\OnehalfSpacing 
%
% Sets numbering division level
\setsecnumdepth{subsection} 
\maxsecnumdepth{subsubsection}
%
% Chapter style (taken and slightly modified from Lars Madsen Memoir Chapter 
% Styles document
\usepackage{calc,soul,fourier}
\makeatletter 
\newlength\dlf@normtxtw 
\setlength\dlf@normtxtw{\textwidth} 
\newsavebox{\feline@chapter} 
\newcommand\feline@chapter@marker[1][4cm]{%
	\sbox\feline@chapter{% 
		\resizebox{!}{#1}{\fboxsep=1pt%
			\colorbox{gray}{\color{white}\thechapter}% 
		}}%
		\rotatebox{90}{% 
			\resizebox{%
				\heightof{\usebox{\feline@chapter}}+\depthof{\usebox{\feline@chapter}}}% 
			{!}{\scshape\so\@chapapp}}\quad%
		\raisebox{\depthof{\usebox{\feline@chapter}}}{\usebox{\feline@chapter}}%
} 
\newcommand\feline@chm[1][4cm]{%
	\sbox\feline@chapter{\feline@chapter@marker[#1]}% 
	\makebox[0pt][c]{% aka \rlap
		\makebox[1cm][r]{\usebox\feline@chapter}%
	}}
\makechapterstyle{daleifmodif}{
	\renewcommand\chapnamefont{\normalfont\Large\scshape\raggedleft\so} 
	\renewcommand\chaptitlefont{\normalfont\Large\bfseries\scshape} 
	\renewcommand\chapternamenum{} \renewcommand\printchaptername{} 
	\renewcommand\printchapternum{\null\hfill\feline@chm[2.5cm]\par} 
	\renewcommand\afterchapternum{\par\vskip\midchapskip} 
	\renewcommand\printchaptertitle[1]{\color{gray}\chaptitlefont\raggedleft ##1\par}
} 
\makeatother 
\chapterstyle{daleifmodif}
%
% UoB guidelines:
%
% The pages should be numbered consecutively at the bottom centre of the
% page.
\makepagestyle{myvf} 
\makeoddfoot{myvf}{}{\thepage}{} 
\makeevenfoot{myvf}{}{\thepage}{} 
\makeheadrule{myvf}{\textwidth}{\normalrulethickness} 
\makeevenhead{myvf}{\small\textsc{\leftmark}}{}{} 
\makeoddhead{myvf}{}{}{\small\textsc{\rightmark}}
\pagestyle{myvf}
%
% Oscar's command (it works):
% Fills blank pages until next odd-numbered page. Used to emulate single-sided
% frontmatter. This will work for title, abstract and declaration. Though the
% contents sections will each start on an odd-numbered page they will
% spill over onto the even-numbered pages if extending beyond one page
% (hopefully, this is ok).
\newcommand{\clearemptydoublepage}{\newpage{\thispagestyle{empty}\cleardoublepage}}
%
%
% Creates indexes for Table of Contents, List of Figures, List of Tables and Index
\makeindex
% \printglossaries below creates a list of abbreviations. \gls and related
% commands are then used throughout the text, so that latex can automatically
% keep track of which abbreviations have already been defined in the text.
%
% The import command enables each chapter tex file to use relative paths when
% accessing supplementary files. For example, to include
% chapters/brewing/images/figure1.png from chapters/brewing/brewing.tex we can
% use
% \includegraphics{images/figure1}
% instead of
% \includegraphics{chapters/brewing/images/figure1}
\usepackage{import}

% Add other packages needed for chapters here. For example:
\usepackage{lipsum}					%Needed to create dummy text
\usepackage{amsfonts} 					%Calls Amer. Math. Soc. (AMS) fonts
\usepackage[centertags]{amsmath}			%Writes maths centred down
\usepackage{stmaryrd}					%New AMS symbols
\usepackage{amssymb}					%Calls AMS symbols
\usepackage{amsthm}					%Calls AMS theorem environment
\usepackage{newlfont}					%Helpful package for fonts and symbols
\usepackage{layouts}					%Layout diagrams
\usepackage{graphicx}					%Calls figure environment
\usepackage{longtable,rotating}			%Long tab environments including rotation. 
\usepackage[applemac]{inputenc}			%Needed to encode non-english characters 
									%directly for mac
\usepackage{colortbl}					%Makes coloured tables
\usepackage{wasysym}					%More math symbols
\usepackage{mathrsfs}					%Even more math symbols
\usepackage{float}						%Helps to place figures, tables, etc. 
\usepackage{verbatim}					%Permits pre-formated text insertion
\usepackage{upgreek }					%Calls other kind of greek alphabet
\usepackage{latexsym}					%Extra symbols
\usepackage[square,numbers,
		     sort&compress]{natbib}		%Calls bibliography commands 
\usepackage{url}						%Supports url commands
\usepackage{etex}						%eTeXÕs extended support for counters
\usepackage{fixltx2e}					%Eliminates some in felicities of the 
									%original LaTeX kernel
\usepackage[spanish,english]{babel}		%For languages characters and hyphenation
\usepackage{color} %Creates coloured text and background
\usepackage[dvipsnames]{xcolor}
\usepackage[colorlinks=true,
		     allcolors=black]{hyperref}              %Creates hyperlinks in cross references
\usepackage{memhfixc}					%Must be used on memoir document 
									%class after hyperref
\usepackage{enumerate}					%For enumeration counter
\usepackage{footnote}					%For footnotes
\usepackage{microtype}					%Makes pdf look better.
\usepackage{rotfloat}					%For rotating and float environments as tables, 
									%figures, etc. 
\usepackage{alltt}						%LaTeX commands are not disabled in 
									%verbatim-like environment
\usepackage[version=0.96]{pgf}			%PGF/TikZ is a tandem of languages for producing vector graphics from a 
\usepackage{tikz}						%geometric/algebraic description.
\usetikzlibrary{arrows,shapes,snakes,
		       automata,backgrounds,
		       petri,topaths}				%To use diverse features from tikz
 \usepackage{array}
\usepackage{tabu}  
%\usepackage{hyperref}
\usepackage{tabularx} % for fancy text filled tables
\usepackage{color}   %May be necessary if you want to color links
\usepackage{hyperref}
\hypersetup{
    colorlinks=true, %set true if you want colored links
    linktoc=all,     %set to all if you want both sections and subsections 
    linkcolor=blue,  %choose some color if you want links to stand out
}
%							
%Reduce widows  (the last line of a paragraph at the start of a page) and orphans 
% (the first line of paragraph at the end of a page)
\widowpenalty=1000
\clubpenalty=1000
%
% New command definitions for my thesis
%
\newcommand{\keywords}[1]{\par\noindent{\small{\bf Keywords:} #1}} %Defines keywords small section
\newcommand{\parcial}[2]{\frac{\partial#1}{\partial#2}}                             %Defines a partial operator
\newcommand{\vectorr}[1]{\mathbf{#1}}                                                        %Defines a bold vector
\newcommand{\vecol}[2]{\left(                                                                         %Defines a column vector
	\begin{array}{c} 
		\displaystyle#1 \\
		\displaystyle#2
	\end{array}\right)}
\newcommand{\mados}[4]{\left(                                                                       %Defines a 2x2 matrix
	\begin{array}{cc}
		\displaystyle#1 &\displaystyle #2 \\
		\displaystyle#3 & \displaystyle#4
	\end{array}\right)}
\newcommand{\pgftextcircled}[1]{                                                                    %Defines encircled text
    \setbox0=\hbox{#1}%
    \dimen0\wd0%
    \divide\dimen0 by 2%
    \begin{tikzpicture}[baseline=(a.base)]%
        \useasboundingbox (-\the\dimen0,0pt) rectangle (\the\dimen0,1pt);
        \node[circle,draw,outer sep=0pt,inner sep=0.1ex] (a) {#1};
    \end{tikzpicture}
}
\newcommand{\range}[1]{\textnormal{range }#1}                                             %Defines range operator
\newcommand{\innerp}[2]{\left\langle#1,#2\right\rangle}                                 %Defines inner product
\newcommand{\prom}[1]{\left\langle#1\right\rangle}                                         %Defines average operator
\newcommand{\tra}[1]{\textnormal{tra} \: #1}                                                       %Defines trace operator
\newcommand{\sign}[1]{\textnormal{sign\,}#1}                                                   %Defines sign operator
\newcommand{\sech}[1]{\textnormal{sech} #1}                                                  %Defines sech
\newcommand{\diag}[1]{\textnormal{diag} #1}                                                    %Defines diag operator
\newcommand{\arcsech}[1]{\textnormal{arcsech} #1}                                       %Defines arcsech
\newcommand{\arctanh}[1]{\textnormal{arctanh} #1}                                         %Defines arctanh
%Change tombstone symbol
\newcommand{\blackged}{\hfill$\blacksquare$}
\newcommand{\whiteged}{\hfill$\square$}
\newcounter{proofcount}
\renewenvironment{proof}[1][\proofname.]{\par
 \ifnum \theproofcount>0 \pushQED{\whiteged} \else \pushQED{\blackged} \fi%
 \refstepcounter{proofcount}
 \normalfont 
 \trivlist
 \item[\hskip\labelsep
       \itshape
   {\bf\em #1}]\ignorespaces
}{%
 \addtocounter{proofcount}{-1}
 \popQED\endtrivlist
}
%
%
% New definition of square root:
% it renames \sqrt as \oldsqrt
\let\oldsqrt\sqrt
% it defines the new \sqrt in terms of the old one
\def\sqrt{\mathpalette\DHLhksqrt}
\def\DHLhksqrt#1#2{%
\setbox0=\hbox{$#1\oldsqrt{#2\,}$}\dimen0=\ht0
\advance\dimen0-0.2\ht0
\setbox2=\hbox{\vrule height\ht0 depth -\dimen0}%
{\box0\lower0.4pt\box2}}
%
% My caption style
\newcommand{\mycaption}[2][\@empty]{
	\captionnamefont{\scshape} 
	\changecaptionwidth
	\captionwidth{0.9\linewidth}
	\captiondelim{.\:} 
	\indentcaption{0.75cm}
	\captionstyle[\centering]{}
	\setlength{\belowcaptionskip}{10pt}
	\ifx \@empty#1 \caption{#2}\else \caption[#1]{#2}
}
%
% My subcaption style
\newcommand{\mysubcaption}[2][\@empty]{
	\subcaptionsize{\small}
	\hangsubcaption
	\subcaptionlabelfont{\rmfamily}
	\sidecapstyle{\raggedright}
	\setlength{\belowcaptionskip}{10pt}
	\ifx \@empty#1 \subcaption{#2}\else \subcaption[#1]{#2}
}
%
%An initial of the very first character of the content
\usepackage{lettrine}
\newcommand{\initial}[1]{%
	\lettrine[lines=3,lhang=0.33,nindent=0em]{
		\color{gray}
     		{\textsc{#1}}}{}}
%
% Theorem styles used in my thesis
%
\theoremstyle{plain}
\newtheorem{theo}{Theorem}[chapter]
\theoremstyle{plain}
\newtheorem{prop}{Proposition}[chapter]
\theoremstyle{plain}
\theoremstyle{definition}
\newtheorem{dfn}{Definition}[chapter]
\theoremstyle{plain}
\newtheorem{lema}{Lemma}[chapter]
\theoremstyle{plain}
\newtheorem{cor}{Corollary}[chapter]
\theoremstyle{plain}
\newtheorem{resu}{Result}[chapter]
%
% Hyphenation for some words
%
\hyphenation{res-pec-tively}
\hyphenation{mono-ti-ca-lly}
\hyphenation{hypo-the-sis}
\hyphenation{para-me-ters}
\hyphenation{sol-va-bi-li-ty}
%
%
\begin{document}
% UoB guidlines:
%
% Preliminary pages
% 
% The five preliminary pages must be the Title Page, Abstract, Dedication
% and Acknowledgements, Author's Declaration and Table of Contents.
% These should be single-sided.
% 
% Table of contents, list of tables and illustrative material
% 
% The table of contents must list, with page numbers, all chapters,
 % sections and subsections, the list of references, bibliography, list of
% abbreviations and appendices. The list of tables and illustrations
% should follow the table of contents, listing with page numbers the
% tables, photographs, diagrams, etc., in the order in which they appear
% in the text.
% 
\frontmatter
\pagenumbering{roman}
%
%
% File: Title.tex
% Author: V?ctor Bre?a-Medina
% Description: Contains the title page
%
% UoB guidelines:
% 
% At the top of the title page, within the margins, the dissertation should give the title and, if 
% necessary, sub-title and volume number. If the dissertation is in a language other than English, the 
% title must be given in that language and in English. The full name of the author should be in the 
% centre of the page. At the bottom centre should be the words ?A dissertation submitted to the 
% University of Bristol in accordance with the requirements for award of the degree of ? in the 
% Faculty of ...?, with the name of the school and month and year of submission. The word count of 
% the dissertation (text only) should be entered at the bottom right-hand side of the page.
%
%
\begin{titlingpage}
\begin{SingleSpace}
\calccentering{\unitlength} 
\begin{adjustwidth*}{\unitlength}{-\unitlength}
%\vspace*{6mm}
\begin{center}
\includegraphics[scale=1.18]{logos/politecnico.jpeg}\\ \vspace{2.5mm}
\vspace*{6mm}
\textsc{\Large School of Industrial and Information Engineering}\\[0.5cm]
\textsc{\Large Master of Science in Management Engineering}\\[0.5cm]
\end{center}
\rule[0.5ex]{\linewidth}{2pt}\vspace*{-\baselineskip}\vspace*{3.2pt}
\rule[0.5ex]{\linewidth}{1pt}\\[\baselineskip]
{\huge Application of Advanced Project Planning Methodologies for a Nuclear Infrastructure Project at the European Organization for Nuclear Research }\\[4mm]
%{\Large \textit{Effects of spatial microscopic defects on luminescence intensity}}\\
\rule[0.5ex]{\linewidth}{1pt}\vspace*{-\baselineskip}\vspace{3.2pt}
\rule[0.5ex]{\linewidth}{2pt}\\
\vspace{1.5mm}
%{\large By}\\


\vspace{2mm}
\begin{flushleft}
{\LARGE \textbf{Academic Supervisor:} Prof. Dr. \textsc{Sergio Terzi}}\\ 
\vspace{2mm}
{\LARGE \textbf{CERN Supervisor:} Dr. \textsc{Luigi Serio}}\\ 
\vspace{2mm}
{\LARGE  \textbf{Master Thesis Author:} \textsc{Deepti Kandhol (896373)}}\\ 
\end{flushleft}
\vspace{6mm}
\begin{center}
	%\includegraphics[scale=0.085]{logos/CERN.png}\\
	\vspace{6mm}
	\large \textbf{Academic Year : 2019 - 2020}
	\vspace{2mm}
\end{center}



\end{adjustwidth*}
\end{SingleSpace}
\end{titlingpage}
\clearemptydoublepage
%
%
% File: abstract.tex
% Author: V?ctor Bre?a-Medina
% Description: Contains the text for thesis abstract
%
% UoB guidelines:
%
% Each copy must include an abstract or summary of the dissertation in not
% more than 300 words, on one side of A4, which should be single-spaced in a
% font size in the range 10 to 12. If the dissertation is in a language other
% than English, an abstract in that language and an abstract in English must
% be included.

\chapter*{\huge Abstract}
\begin{SingleSpace}
%\initial{A}
\emph{Application of advanced planning methodologies for a nuclear infrastructure project at CERN. }
\bigskip
The European Center for Nuclear Research is an International research organization that operates the largest particle physics laboratory in the world. CERN main function is to provide the particle accelerators and other infrastructure needed for high-energy physics research and as a result numerous experiments have been constructed at CERN through international collaborations.

 A Nuclear Infrastructure project, the Nano-Lab, is currently in his design phase. It is an extension of an existing building and will be comprised of a radioactive material storage area and two laboratories for manipulating uranium nanoparticles. The project manager assigned to the Nano-lab project is responsible for initiating, planning, executing, controlling and closing projects. The Nano-lab project planning since project is critical in nature owing to the complex scientific environment at CERN and the potential risks associated to it. 

The aim of this work is to analyze the classical planning tools employed for this project, review their limitations and drawbacks, study and implement alternative and advanced planning methodologies. 

The conventional tool involves identifying the project activities, estimating the activity duration, determine the necessary resources i.e. cost. Gantt chart is used as the main tool for development of a documented project plan. The key point of analysis is the estimates made for time and cost for Nano-lab project. Estimates are made based on historical data form other successfully implemented projects, professional experience – Technical and managerial expertise. 

The potential risks associated with Nano-lab project are related to quality, cost and schedule with an impact on use and coordination of activities and resources. The traditional project planning methodologies puts special emphasis on linear processes, upfront planning and prioritization. As per classical methods like WBS, Gantt chart, time and budget are fixed, and requirements are variable due to which it often faces budget and timeline issues. The classical methods are not able to interpret the uncertainty with the activities done in parallel by different cross functional teams in case of involving multiple stakeholders, hence any delay in work can lead to additional complexities and will have impact on both time and budget. All these risks cannot be reflected upon the Gantt chart which becomes difficult to monitor and control. 

Use of Advanced planning methodologies like Monte-Carlo can address these uncertainties that poses potential risks for the Nano-lab project. Monte -Carlo simulation is used in case of complex estimation scenario that involves high degree of complexity and uncertainty to analyze the likelihood of meeting the objectives.
Monte-Carlo simulation was conducted for Nano-lab project for total time and cost and to determine the likelihood of meeting the objectives in order to assess risks and implement where mitigation actions are required. Critical path is determined and considered for further calculations as activities are in parallel as per project planning. The results display the likelihood of completion of the project of nano-lab project for both time and cost. The results are validated by comparing them with the traditional planning estimates by the project manager, the variations can be used to review and monitoring the planning and scheduling objectives and helps in making informed project decisions. 

Monte-Carlo is useful to find likelihood of meeting project milestone and immediate goals. It can also predict the likelihood of schedule and cost overruns, hence efficient in adapting to the changes by tracking the immediate goals and the milestones of the project. Further, sensitivity and uncertainty analysis can be done for risk variables of the project. Monte-Carlo is a valuable advanced planning methodology as the results are validated with the actual estimates for schedule and cost and the activities performance can be tracked and it’s easy to revise schedule accordingly. It is concluded that a combination of both classical tools and Monte-Carlo proves to be an incredible technique to govern processes and control variables for project management.
 
\end{SingleSpace}
\clearpage
\clearemptydoublepage
%
%
% file: dedication.tex
% author: V?ctor Bre?a-Medina
% description: Contains the text for thesis dedication
%

\chapter*{Dedication and acknowledgments}
\begin{SingleSpace}
%\section*{Acknowledgments}
First and foremost, I would like to thank my both supervisors Prof. Sergio Terzi from Politecnico Di Milano and Dr. Luigi Serio at European Council for Nuclear Research CERN. 

I am grateful to Professor Sergio Terzi who agreed to supervise and guide me throughout. This accomplishment would not have been possible without his academic and moral support. I am also grateful with the bottom of my heart for the opportunity to be a part of Dr.Luigi Serio administration, resources and performance group in engineering department at CERN. Taking this as an opportunity, I want to show my profound gratitude to everyone who has facilitated me to fulfill this task within the deadlines. 

Most importantly, I want to acknowledge European Council for Nuclear Research and especially my CERN supervisor, Dr. Luigi Serio.  I would like to thank him and share my deepest gratitude for the time he took to discuss my work, providing me with deep in insights into the Project management for complex projects like nuclear infrastructure. He pushed me to work independently and always provided his counsel whenever I needed it. I have learned a great deal working with him and always got inspired by his own example to work hard and expand my knowledge and thinking. His optimism and kindness motivates me to keep searching for greater heights.

My thesis would not have been possible without the amiable support of two of my good friends Gul Muhammad and Keshava Prasad Gubbi and for their fruitful discussions about writing and organizing my thesis. 

I am eternally grateful to my Father and Siblings who have always encouraged me to pursue my dreams. They have always believed in me and are my pillars of support.

I would like to dedicate this work to my life-coach my Late mother Bimla Kandhol, who had always encouraged me to achieve growth and success in life. I always miss your voice and presence. 

\end{SingleSpace}
\clearpage
\clearemptydoublepage
%
%
% File: declaration.tex
% Author: V?ctor Bre?a-Medina
% Description: Contains the declaration page
%
% UoB guidelines:
%
% Author's declaration
%
% I declare that the work in this dissertation was carried out in accordance
% with the requirements of the University's Regulations and Code of Practice
% for Research Degree Programmes and that it has not been submitted for any
% other academic award. Except where indicated by specific reference in the
% text, the work is the candidate's own work. Work done in collaboration with,
% or with the assistance of, others, is indicated as such. Any views expressed
% in the dissertation are those of the author.
%
% SIGNED: .............................................................
% DATE:..........................
%
\chapter*{Note of Confidentiality}
\begin{SingleSpace}
\begin{quote}
This thesis is based on internal, confidential data of information on CERN - European Center for Nuclear Excellence, Geneva.
This work may be available to first and second reviewers and authorized members of the board examiners.
Any publication and duplication of this work is prohibited. Any inspection by the third parties requires the expressed permission of the author and the organization.

%declare that the work in this dissertation was carried out in accordance with the requirements of  the University's Regulations and Code of Practice for Research Degree Programmes and that it  has not been submitted for any other academic award. Except where indicated by specific  reference in the text, the work is the candidate's own work. Work done in collaboration with, or with the assistance of, others, is indicated as such. Any views expressed in the dissertation are those of the author.

\vspace{1.5cm}
\noindent
%\hspace{-0.75cm}\textsc{SIGNED: .................................................... DATE: ..........................................}
\end{quote}
\end{SingleSpace}
\clearpage
\clearemptydoublepage
%

\renewcommand{\contentsname}{Table of Contents}
\maxtocdepth{subsection}
\tableofcontents*
\addtocontents{toc}{\par\nobreak \mbox{}\hfill{\bf Page}\par\nobreak}
\clearemptydoublepage
%============================================
%\listoftables
%\addtocontents{lot}{\par\nobreak\textbf{{\scshape Table} \hfill Page}\par\nobreak}
%\clearemptydoublepage
%============================================
\listoffigures
\addtocontents{lof}{\par\nobreak\textbf{{\scshape Figure} \hfill Page}\par\nobreak}
\clearemptydoublepage
%================================================
%
% The bulk of the document is delegated to these chapter files in
% subdirectories.
\mainmatter
%
\import{chapters/chapter01/}{chap01.tex}
\clearemptydoublepage
%\clearpage
\import{chapters/chapter02/}{chap02.tex}
\clearemptydoublepage
\import{chapters/chapter03/}{chap03.tex}
\clearemptydoublepage
\import{chapters/chapter04/}{chap04.tex}
\clearemptydoublepage
\import{chapters/chapter05/}{chap05.tex}
\clearemptydoublepage
%
% And the appendix goes here
\appendix
\import{chapters/appendices/}{app0A.tex}
\clearemptydoublepage
\import{chapters/appendices/}{app0B.tex}
\clearemptydoublepage
\import{chapters/appendices/}{app0C.tex}
% Apparently the guidelines don't say anything about citations or
% bibliography styles so I guess we can use anything.
\backmatter
\bibliographystyle{siam}
\refstepcounter{chapter}
\bibliography{thesisbiblio}
\clearemptydoublepage
%
% Add index
%\printindex
%   
\end{document}