\let\textcircled=\pgftextcircled
\chapter{Research Methodology}
\label{chap:RM}

We denote a generic variable "v" and calculate its likelihood $p(t)$ called Probability. These quantities can be extended to calculation of time and cost estimations for execution of the project. 

\subsection{A: Likelihood to Probability}
For a, b being the range of the activity duration, such that $v < a$ and $v > b$,

\begin{equation}
	p(v)  = 0 
\end{equation}

and for  a $\leqslant$ v $\leqslant$ b,

\begin{equation}
\sum_{v} p(v) = 1
\end{equation}


\subsection{B: Triangular Distribution}

3 points, minimum, most likely and maximum.

To calculate the probability associated to each variable 


\subsection{C: Area under Triangle}

The area under the triangle must be equal to 1, as the time taken to complete the task between $ v_{min} $ and $ v_{max} $ and hence 

\begin{equation}
\frac{1}{2} * base * altitude = \frac{1}{2} * (v_{max} - v_{min} ) * p(v_{ml}) = 1
\label{eqn3}
\end{equation}

where $p(v_{ml} $ is the probability of the variable ml.

Furthermore,


\begin{equation}
p(v_{ml}) = \frac{2}{  (v_{max} - v_{min})} 
\label{eqn4}
\end{equation}

To derive probability for any quantity (v) lying between $ v_{min} $ and $ v_{ml} $ 

\begin{equation}
\frac{(v - v_{min})}{ p(t) }  = \frac{(v_{ml} - v_{min})}{ p(ml) } 
\label{eqn5}
\end{equation}

Consequence of ratios on either side of \ref{eqn5} is equal to slope of line joining points $(v_{min}, 0)$ and $( v_{ml}, p(ml) )$


\subsection{D: }
Substituting \ref{eqn3} in \ref{eqn4} and simplifying,

\begin{equation}
p(t)  = \frac{2* (v - v_{min})}{ (v_{ml} - v_{min}) (v_{max} - v_{min}) } 
\label{eqn6}
\end{equation}

for $ v_{min} \leqslant v \leqslant v_{ml}$.

Similarly, probability for time between $(v_{ml}$ and $ v_{max})$,

\begin{equation}
p(t)  = \frac{2* (v_{min} - v)}{ (v_{max} - v_{ml}) (v_{max} - v_{min}) } 
\label{eqn7}
\end{equation}

for $ v_{ml} \leqslant v \leqslant v_{max}$.

Equations \ref{eqn6} and \ref{eqn7} provides Probability distribution Functions (PDF) for times existing between $v_{min}$ and $v_{max}$. 


In Monte Carlo simulations, the PDF's are not used rather Cumulative Distribution Functions (CDF) is used, with probability (p(t)) that is computed as a function of time (t).

To reiterate, the PDF quantifies the completion probability $p(t)_{PDF}$ of a task at time (t), whereas $p(t)_{CDF}$ indicates the probability of task completion by time (t).

The CDF and $p(t)_{CDF}$ is essentially sum of all probabilities between $v_{min}$ and $v$. 


For $ v < v_{min}$, area under triangle with vertices at $(v_{min}, 0)$, $(v, 0)$ and $(v, v(t))$ indicates probability.

Using triangle formula and Equation \ref{eqn6}, 

\begin{equation}
p(t)  = \frac{2* (v - v_{min})^{2}}{ (v_{ml} - v_{min}) (v_{max} - v_{min}) } 
\label{eqn8}
\end{equation}

for $ v_{min} \leqslant v \leqslant v_{ml}$.

For the case of $ v \geqslant v_{ml}$, the area under curve equals the total area subtracted by the area enclosed by triangle between $v$ and $v_{max}$,

\begin{equation}
p(t)  = 1 -  \frac{(v_{max} - v)^{2}}{ (v_{ml} - v_{min}) (v_{max} - v_{min}) } 
\label{eqn9}
\end{equation}

For $ v_{ml} \leqslant v \leqslant v_{max}$, $p(t)$ ranges between 0 and $v_{min}$ and increases monotonically, attaining a value of 1 at $v_{max}$. 

Put the ranges and $ a = v_{min}, b = v_{ml}, c = v_{max}$, the resulting PDF and CDF would be 





To simulate using Monte Carlo generate a random number between 0 and 1, it corresponds to the probability that the task will finish until time (t).

Find the time t, which corresponds to value of probability, this is the completion time for this Simulation Iteration. Incidentally, this method is called Increase Transform Sampling. 

Solving equations \ref{eqn8} and  \ref{eqn9} yields the following expression for time (t):

 \begin{equation}
 v = v_{min} + \sqrt{ p(v) (v_{ml} - v_{min}) * (v_{max} - v_{min})}
 \label{eqn10}
 \end{equation}
 
for $ v_{min} \leqslant v \leqslant v_{ml}$ and 

 \begin{equation}
v = v_{max} - \sqrt{1 - (p(v) (v_{max} - v_{ml}) * (v_{max} - v_{min}))}
\label{eqn10}
\end{equation}

for $ v_{ml} \leqslant v \leqslant v_{max}$. 

The probability for $v_{min}$ is 0 and for $v_{max}$ is 1. 

The $P(v_{ml})$ can be calculated by using equation \ref{eqn8} wherein $v = v_{max}$ reduces the probability to 

\begin{equation}
p(v)_{ml}  = \frac{(v_{ml} - v_{min})}{  (v_{max} - v_{min}) } 
\label{eqn11}
\end{equation}