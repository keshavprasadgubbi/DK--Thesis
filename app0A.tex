\chapter{Appendix A}
\label{app:app01}
\section{Thin film solar cells}
Photovoltaic is a multidisciplinary field that requires an understanding of physics, chemistry, material science and production technologies. People who dedicated their life in this field have three common objectives (1) to improve the conversion efficiency of the devices, (2) to develop processes that will allow for lower production costs, (3) to ensure that module performances will be maintained for several decades in outdoor conditions, thereby providing much more energy than used in production. Of recent people have realized that thin  film (TF) technilogy is a way to achieve the above objectives. TF technologies use small amounts of active materials and can be manufactured at a lower full cost than c-Si. They have short energy pay-back times (i.e. <1yr in southern Europe) despite their lower efficiency, good stability and lifetimes comparable to c-Si modules. 
\\
\\Thin-film photovoltaics has been defined in different context.% The definition given by Chopra et al. \cite{thinfilm} provides a good starting point and also yields a criterion to discriminate the term 'thin film' from 'thick film'. They define a thin film as a material ‘created ab initio by the random nucleation and growth processes of individually condensing/reacting atomic/ionic/molecular species on a substrate. The structural, chemical, metallurgical and physical properties of such a material are strongly dependent on a large number of deposition parameters and may also be thickness dependent. 
The definition from \cite{Martevs},Thin-film (TF) photovoltaics (PV) is based on large-area planar devices where conversion of light into electricity takes place in a thin multilayer structure only a few micrometres thick. Thin films may encompass a considerable thickness range, varying from a few nanometers to tens of micrometers and could also be  defined in terms of the production processes rather than by thickness.\\
On the physics point view Thin film defined from other conventional semiconductors such as crystalline silicon which has an indirect band gap with weak absorption,Thin films have a direct band gap hence strong absorption properties. 
\subsection{Thin film absorber materials}
Conventionally, photovoltaic materials use inorganic semiconductors. The semiconductors of interest allow the formation of charge-carrier separating junctions. The junction can be either a homojunction (like in Si) or a heterojunction with other materials to collect the excess carriers when exposed to light. In principle, a large number of semiconductor materials are eligible, but only a few of them are of sufficient interest. Ideally, the absorber material of an
efficient terrestrial solar cell should be a semiconductor with a bandgap of 1-1.5 eV with a high
solar optical absorption in the wavelength region of 350-1000 nm, a high
quantum yield for the excited carriers, a long diffusion length low recombination velocity. If
all these constraints are satisfied and the basic material is widely available, the material allows
in principle the manufacturing of a thin-film solar cell device.
From the point of view of processing, manufacturing and reproducibility, elemental semiconductors
provide a simple and straightforward approach to manufacture thin-film solar cells.
Amongst TF PV devices are amorphous-silicon (a-Si), multi-junction amorphous-silicon/microcrys-\\talline-silicon $(a-Si/\mu c-Si)$, chalcopyrite $Cu(In_x,Ga_{(1-x))} (Se,S)_2 (CIGS)$, and cadmium telluride solar cells (CdTe), as well as dye-sensitised solar cells (DSSC), organic solar cells (OSC) and as the youngest type of TF solar cells, the so-called perovskite solar cells.\\
III-V compound materials like GaAs, InP and their derived alloys and compounds, which
most often have a direct bandgap character, are ideal for photovoltaic applications, but are far too
expensive for large-scale commercial applications, because of the high cost of the necessary precursors for the deposition and the deposition systems itself. \\
More appealing from the point of view of ease of processability and cost of material and
deposition are the 'II-VI compound materials' like CdTe or variations on this like CuInSe$_2$. The interest in these materials for thin-film solar cell manufacturing is essentially based on two elements. Because of the chemical structure of these materials the internal (grain boundaries,
interfaces) and external surfaces are intrinsically well passivated and characterized by a low recombination velocity for excess carriers. The low recombination activity at the grain boundaries allows high solar cell efficiencies even when the material is polycrystalline with grain sizes in the order of only a few $\mu m$. This is to be contrasted with crystalline Si where grain boundaries are normally characterized by a high recombination velocity. Moreover, the polycrystallinity allows a large number of substrates (glass, steel foil, . . . ) and is compatible with low-cost temperature deposition techniques, as there is no need for epitaxial growth or high temperatures to obtain large grain sizes. A second important property is the possibility to tailor the bandgap e.g. replacing Se by S in CuInSe$_2$ results in a material with a higher bandgap.
This property allows one to build in bandgap gradients aiding the collection of excess carriers and, ultimately, could even be used to develop multijunction solar cells. 
\\
\\
A new record for thin film solar cell  efficiency of  22.6\% has been achieved by ZSW\footnote{ZSW: https://www.zsw-bw.de/en.html} (Zentrum fur Sonnenenergie- und Wasserstoff-Forschung - or Center for Solar Energy and Hydrogen Research -Baden-Wurttemberg) in Stuttgart, Germany. ZSW's latest record-setting cell has an area of about 0.5$cm^2$ using CIGS technology\cite{ZSW} beating the the Japanese firm Solar Frontier\footnote{Solar Frontier: http://www.solar-frontier.com/eng/}, which exceeded 22.3\% efficiency using CIS technology\cite{frontier}.
The  world conversion efficiency record of 22.1\% for cadmium-telluride solar photovoltaic cells has been achieved by First Solar\footnote{\href{here}{First Solar: http://www.firstsolar.com}}\cite{Firstsolar}. A world record initial efficiency of 14.8\% has been achieved in tandem thin-film silicon solar cells\cite{silicon}. The youngest TF solar cells, (perovskites) have shown fast improvement with a  record power-conversion efficiency of 21.1\% and 18\% after 250 hours under operational conditions\cite{perovskite}.
\\
\\Despite more than 30 years of research invested in each of the thin-film solar cell technologies considered here, a large series of questions has still to be answered. The need for more 'know-why' in addition to the available 'know-how' is urged by the responsibility of scientists toward a steadily growing industry and toward a world in need for clean energy. Fortunately, more and more specialists for sophisticated physical and chemical analysis methods enter the field and help improving our common understanding as well as improving our technology. The most satisfying answers always will arise from a combination of a solid understanding of the photovoltaic principles with the results from various methods analyzing the electronic, chemical, and structural properties of all the layers and interfaces in the device.