%
% File: abstract.tex
% Author: V?ctor Bre?a-Medina
% Description: Contains the text for thesis abstract
%
% UoB guidelines:
%
% Each copy must include an abstract or summary of the dissertation in not
% more than 300 words, on one side of A4, which should be single-spaced in a
% font size in the range 10 to 12. If the dissertation is in a language other
% than English, an abstract in that language and an abstract in English must
% be included.

\chapter*{\huge Abstract}
\begin{SingleSpace}
%\initial{A}
\emph{Application of advanced planning methodologies for a nuclear infrastructure project at CERN. }
\bigskip


The European Organization for Nuclear Research is an international research organization that operates the largest particle physics laboratory in the world. CERN supports the high-energy physics research by providing the particle accelerators and other infrastructure through international collaboration; several experiments have been built at CERN through these multi-national collaborations.

A nuclear infrastructure project, the Nano-Lab is currently in its design phase. It is an extension of an existing building and will be comprised of a radioactive material storage area and two laboratories for manipulating uranium nano-particles. The project manager assigned to the Nano-lab project is responsible for initiating, planning, executing, controlling, and closing projects. Planning is the most crucial step of the project life cycle for a critical project like Nano-Lab. The risks posed can be attributed to the critical nature owing to the complex scientific environment at CERN.

The aim of this work is to analyze the classical planning tools employed for this project, review their limitations and drawbacks, study and implement alternative and advanced planning methodologies. The conventional tool involves identifying the project activities, estimating the activity duration, determine the necessary resources i.e. cost. Gantt chart is used as the main tool for the development of a documented project plan. The key point of the analysis is the estimates made for time and cost for the Nano-Lab project. Estimates are made based on historical data from other successfully implemented projects, professional experience, technical and managerial expertise.

The potential risks associated with the Nano-Lab project are related to quality, cost, and schedule with an impact on the use and coordination of activities and resources. The traditional project planning methodologies puts special emphasis on linear processes, upfront planning, and prioritization. As per classical methods like WBS, Gantt chart, time and budget are fixed, with any variability in the project requirements due to any circumstances often leads to budget and timeline issues. The classical methods are not able to interpret the uncertainty with the activities done in parallel by different cross-functional teams in case of involving multiple stakeholders, hence any delay in work can lead to additional complexities and will have an impact on both time and budget. It is to be noted that not all these risks can be reflected upon the Gantt chart, which becomes difficult to monitor and control.

Use of Advanced planning methodologies like Monte-Carlo can address these uncertainties that pose potential risks for the Nano-lab project. Monte-Carlo simulation is used in case of a complex estimation scenario that involves a high degree of complexity and uncertainty to analyze the likelihood of meeting the objectives.

Monte-Carlo simulation was conducted for the Nano-Lab project for total time and cost in order to determine the likelihood of meeting the objectives in order complete the project. A network diagram is created considering the overlaps due to parallel activities for project planning. The result displays the likelihood of completion in terms of probability along with values of final completion-variables in terms of time and cost. The results are validated by comparing them with the traditional planning estimates by the project manager, the variations can be used to review and monitoring the planning and scheduling objectives and helps in making informed project decisions.

The Monte-Carlo simulations results - as mentioned in chapter 6 - explicitly explained the uncertainty related within the probability distribution. The uncertainty is explained through the variables in this case as probability changes rapidly with respect to time or cost. The Monte-Carlo simulation tool in MS-Excel is easily accessible and self-explanatory, which can be easily used by both technical and non-technical team members. 

Furthermore, an extension to the advanced project planning methodologies like Monte-Carlo, uncertainty analysis can be employed to do risk analysis on the identified critical activities and schedule can be updated accordingly keeping in mind th project duration constraints. Monte-Carlo supports efficient project planning and scheduling along with traditional project planning methodologies based on historical estimates. It will not replace human capabilities in making decisions based on experience but an intelligent combination of the two, has the potential to comprehensively address all aspects of project planning and scheduling risks in the Nano-Lab project, CERN and also in similar large-scale, complex projects in other scientific environments.



\end{SingleSpace}
\clearpage