%
% File: abstract.tex
% Author: V?ctor Bre?a-Medina
% Description: Contains the text for thesis abstract
%
% UoB guidelines:
%
% Each copy must include an abstract or summary of the dissertation in not
% more than 300 words, on one side of A4, which should be single-spaced in a
% font size in the range 10 to 12. If the dissertation is in a language other
% than English, an abstract in that language and an abstract in English must
% be included.

\chapter*{\huge Abstract}
\begin{SingleSpace}
%\initial{A}
\emph{Application of advanced planning methodologies for a nuclear infrastructure project at CERN. }
\bigskip


The European Council for Nuclear Research is an international research organization that operates the largest particle physics laboratory in the world. CERN, main function is to provide the particle accelerators and other infrastructure needed for high-energy physics research and as a result numerous experiments have been constructed at CERN through international collaborations. 

A Nuclear Infrastructure project, the Nano-Lab is currently in his design phase. It is an extension of an existing building and will be comprised of a radioactive material storage area and two laboratories for manipulating uranium nano-particles. The project manager assigned to the Nano-lab project is responsible for initiating, planning, executing, controlling, and closing projects. The Nano-lab project planning since the project is critical in nature owing to the complex scientific environment at CERN and the potential risks associated with it.



The aim of this work is to analyze the classical planning tools employed for this project, review their limitations and drawbacks, study and implement alternative and advanced planning methodologies. The conventional tool involves identifying the project activities, estimating the activity duration, determine the necessary resources i.e. cost. Gantt chart is used as the main tool for the development of a documented project plan. The key point of the analysis is the estimates made for time and cost for the Nano-Lab project. Estimates are made based on historical data form other successfully implemented projects, professional experience – Technical and managerial expertise.



The potential risks associated with the Nano-Lab project are related to quality, cost, and schedule with an impact on the use and coordination of activities and resources. The traditional project planning methodologies puts special emphasis on linear processes, upfront planning, and prioritization. As per classical methods like WBS, Gantt chart, time, and budget are fixed, and requirements are variable due to which it often faces budget and timeline issues. The classical methods are not able to interpret the uncertainty with the activities done in parallel by different cross-functional teams in case of involving multiple stakeholders, hence any delay in work can lead to additional complexities and will have an impact on both time and budget. All these risks cannot be reflected upon the Gantt chart which becomes difficult to monitor and control.



Use of Advanced planning methodologies like Monte-Carlo can address these uncertainties that pose potential risks for the Nano-lab project. Monte -Carlo simulation is used in case of a complex estimation scenario that involves a high degree of complexity and uncertainty to analyze the likelihood of meeting the objectives.

Monte-Carlo simulation was conducted for the Nano-Lab project for total time and cost and to determine the likelihood of meeting the objectives in order to assess risks and implement where mitigation actions are required. A network diagram is created considering the overlaps due to parallel activities for project planning. The results displays likelihood of completion in terms of probability along with values of final completion-variables in terms of time and cost. The results are validated by comparing them with the traditional planning estimates by the project manager, the variations can be used to review and monitoring the planning and scheduling objectives and helps in making informed project decisions.

Monte-Carlo is useful to find the likelihood of meeting project milestones and immediate goals. It can also predict the likelihood of schedule and cost overruns, hence efficient in adapting to the changes by tracking the immediate goals and the milestones of the project. Further, sensitivity and uncertainty analysis can be done for risk variables of the project along with Earned Value Management. 

Monte-Carlo is a valuable advanced planning methodology as the results are validated with the actual estimates for schedule and cost and the activities performance can be tracked and it's easy to revise schedule accordingly. It is concluded that a combination of both classical tools and Monte-Carlo proves to be an incredible technique to govern processes and control variables for project management.
 
\end{SingleSpace}
\clearpage