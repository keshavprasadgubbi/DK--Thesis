\let\textcircled=\pgftextcircled
\chapter{Case Study of Nuclear Infrastructure}
\label{chap:chap5}
The Isotope mass Separator On-Line facility (ISOLDE) at CERN is a unique source of low-energy beams of radioactive nuclides, those with too many or too few neutrons to be stable. The facility fulfills in fact the old alchemical dream of changing one element into another. It permits the study of the vast territory of atomic nuclei, including the most exotic species. 

Owing to the volatility and pyrophoric properties of the radioactive particles it is required to have dedicated laboratories to ensure safe handling of the radioactive particles. It is proposed to have an extension to existing Class A laboratory in order to provide safe space for handling and production of radioactive targets by building a dedicated space for carrying out the activities.

This decision for construction of Nano-lab and also benefitting from the already available infrastructure of the existing building.  A nuclear Infrastructure project, the Nano-Lab, is studied during its design phase and currently it is in its execution phase. It is an extension of an existing building and will be comprised of a radioactive material storage area and two laboratories for manipulating uranium nano particles. The Nano-lab project planning is the most critical step of the Project management life cycle owing to the complex scientific environment at CERN and the potential risks associated to it. 

\section{Description of the project}

\textbf{Nano-lab :}  A Nuclear Infrastructure, a  22 m X 7 m building with a floor area of approximately 140 m 2. All reinforced concrete including a part of high density concrete walls (> 3.9 tons / m3)

It is an extension of existing building. The future building is composed of three sections:

\begin{enumerate}
	\item \textbf{Production Laboratory:} A dedicated laboratory for the production radioactive nanoparticles used as targets for ISOLDE facility.
	\item \textbf{Pump stand Laboratory :} Another laboratory  used for manipulation and calibration of the radioactive nano particles.
	\item \textbf{Buffer Area:} A temporary storage area for radioactive waste and will be built of concrete possessing reinforced shielding properties. It will be used as an immediate repository for radioactive waste and also a storage place for accommodating contaminated equipment from the ISOLDE experiment.
\end{enumerate}

\begin{figure}
	\centering
	\includegraphics[height=0.43\textheight]{fig05/Fig1Ch1-TripleConstraint.png}
	\mycaption[ch5.fig1]{Proposed lay-out of the Nanolab. Image Source:}
\end{figure}

In design phase of the project areas, the approximate areas are: 

\begin{enumerate}
	\item Buffer area of $49m^{2}$
	\item Pump stand Laboratory of $23m^{2}$
	\item Production Laboratory of $44m^{2}$
\end{enumerate}

\section {Forecasted planning and scheduling phase} 

Forecasting and the planning phase is carried out by the Project manager assigned by the Project Management team.
The project manager assigned to the Nano-lab project is responsible for initiating, planning, executing, controlling and closing projects. The aim of this work is to analyze the classical planning tools employed for this project, review their limitations and drawbacks, study and implement alternative and advanced planning methodologies. 

Project management, is the application of knowledge, skills, tools, and techniques to project activities to meet requirements.

Project management processes fall into five groups:
\begin{itemize}
	\item  Initiating
	\item  Planning
	\item  Executing
	\item  Monitoring and Controlling
	\item  Closing
	
\end{itemize}


Reference : PMI's A Guide to the Project Management Body of Knowledge (PMBOK Guide)

Project management knowledge draws on ten areas:
\begin{itemize}
	\item Scope
	\item \textbf{Time}
	\item \textbf{Cost}
	\item  Quality
	\item Human resources
	\item Communications
	\item Risk management
	\item Stakeholder management
	\item Integration
\end{itemize}

A work breakdown structure (WBS) in project management and systems engineering is a deliverable oriented breakdown of a project into smaller components. For the Nano-lab project the key project deliverables were dividing into ten work packages.  

The WBS for the Nano-lab displays the work packages involved and then activities are defined accordingly.

Several Work Packages outlines from the WBS are:

\begin{enumerate}
	\item \textbf{Work Package$\_$1$\_$SMB:} Site Management and Building department, they carry out the civil engineering activities strategy, design and studies and further on-site execution works integration with services provided by other work packages. The major activities they will carry out are: 
	\begin{itemize}
		\item 	To conduct the studies - Preliminary Geotechnical study,  design and technical studies 
		
		\item Procurement activities 
		  
		\item Preliminary works on-site
		
		\item To execute activities such as earthwork and concrete, finishing’s
		
		\item Non-IT works in the Nano-lab site 
		
		\item Support with safety  coordinator
	\end{itemize}
	
	\begin{figure}
		\centering
		\includegraphics[height=0.33\textheight]{fig05/risk.png}
		\mycaption[ch5.fig1]{Proposed lay-out of the Nanolab. Image Source:}
	\end{figure}

	\item \textbf{Work Package$\_$2$\_$CV:} Cooling and Ventilation work packages is responsible for the design, construction, commissioning, maintenance and operations of the technical installations in Nano lab.
	The major activities they will carry out are: 
	\begin{itemize}
		\item Design documentation as-built and marking
		
		\item Modifications in CV system in existing building 
		
		\item Ventilation supply and extraction
		
		\item To execute activities such as earthwork and concrete, finishing’s
		
		\item Electrical works, instrumentation and control
		
		\item Compression air works 
		
		\item Pump stand, water cooling, raw water and leak detection 
		
		\item Test and commissioning 
		
		\item Process extraction and fans consolidation
	\end{itemize}
	
	\item \textbf{Work Package$\_$3$\_$Process Equipment:}They are the users of the Nano laboratory who carry out the activities for particles production and carry out responsibilities of target stations and Class A type laboratories.
	The major activities they will carry out are:

	\begin{itemize}
		\item Moving Pump stands
		\item Delivery and installation of equipment in production laboratory 
			\item Conduct equipment testing
		
	\end{itemize}

	\item \textbf{Work Package$\_$4$\_$RP:} Radiation Protection group carry out activities and number of services aim at personnel protection, waste processing, radiation and stray monitoring and also establishes rules and procedure with respect to the radiation protection.
	The major activities they will carry out are:
	\begin{itemize}
		\item upply of monitors for RP.
		\item Installation of Monitors
		\item Conducting Tests

	\end{itemize}
	
	\item \textbf{Work Package$\_$5$\_$EL:} Electrical group provides all electrical supplies, routings and installation, commissioning, and integration activities for electrical services.
	The major activities they will carry out are:
	\begin{itemize}
		\item Installation support
		\item Cable pulling’s for pump stand and fire detection
		\item Installation of luminaires
		\item Chute and catch chute installations
		\item fire cable tests
		\item Excavation of soil for Earthing services	
	\end{itemize}
	
	\item \textbf{Work Package$\_$6$\_$IT:} nformation Technology group provides all IT related services including supplies, installation and integration support to other work packages.
	The major activities they will carry out are: 
	
	\begin{itemize}
		\item Moving GSM cable
		\item IT cable pulling
		\item 	Works related to ducts and plug boxes
		\item Conduct Tests

	\end{itemize}
	
	\item \textbf{Work Package$\_$7$\_$FD:} Fire detection provides the services for fire detection points, routings and procedures and safety aspects for the Nano lab protection for both personnel and equipment.
	The major activities they will carry out are.
	
	\begin{itemize}
		\item Installation of equipment and sniffers
		\item Conduct Tests
	\end{itemize}
	
	\item \textbf{Work Package$\_$8$\_$GAS:} Gas work package takes care of associated technical services for projects by providing gas supply and installation services.
	The major activities they will carry out are:
	
	\begin{itemize}
		\item Installation of gas pipelines for Argon. 
		\item  External work + tests
	\end{itemize}
	
	\item \textbf{Work Package$\_$9$\_$AC:} ccess Control package provides the access control services for the Nano lab to assist the users to ensure access to work in a controlled radioactive environment.
	The major activities they will carry out are: 
	
	\begin{itemize}
		\item Cable pull to door entry
		\item Installation equipment and door connection
		\item Conduct Tests
	\end{itemize}
	
	\item \textbf{Work Package$\_$10$\_$PPM:} Project management team providing coordination, integration and monitoring activities and providing technical and administration support.
	
	
\end{enumerate}


\section{KPI for the project - Time and Cost}

Using WBS, project planning is done for both time and cost KPIs.

Historical data, technical and professional expertise is the foundation for the forecasts related to the time and costs. 

Project is divided into activities which are both in series and parallel.
Planning is depicted via Gantt chart, a classical traditional project management tool for project planning \& scheduling.

Duration of the project: Oct 2019 -June 2021approx. 615 days. 


\begin{figure}
	\centering
	\includegraphics[height=0.43\textheight]{fig05/Fig1Ch1-TripleConstraint.png}
	\mycaption[ch5.fig1]{Gantt chart Schedule Image Source:}
\end{figure}

But uncertainty comes into play, so rigorous governance for processes and control over data is needed to enhance reliability. Use of techniques like Monte-Carlo which is a powerful statistical analysis tool is used to understand the impact of risk and uncertainty in prediction and forecasting models.

\subsection{Use of Monte-Carlo Simulation Model:}

\subsubsection*{Scope:}
Use PERT like approach wherein estimates for an activity are based on 3 points in variables like time and cost for Nanolab project. This three-point estimate will be the starting point for Monte-Carlo simulation. Monte- Carlo simulation provides a principled and intuitive way to obtain probabilistic estimates at the level of an entire project based on estimates of the individual tasks that comprise it.



\begin{center}
	\begin{tabular}{ |c|c|c|c| c|} 
		\hline
		Activity & Minimum & Most Likely & Maximum\\
		\hline
		1 & &  & \\ 
			\hline
		2 & & &\\ 
			\hline
		3& & &\\ 
		\hline
	\end{tabular}
\end{center}

\begin{figure}
	\centering
	\includegraphics[height=0.43\textheight]{fig05/Fig1Ch1-TripleConstraint.png}
	\mycaption[ch5.fig1]{Monte Carlo Flowchart image . Image Source:}
\end{figure}

\subsection{Input data for the model:} 
	
The inputs are:
\begin{itemize}
	\item Variables (v): Time in days \& Cost in kCHF for minimum, most likely and maximum points.
	\item $v\_min$ - minimum time/cost needed to complete the activity,
	\item $v\_ml$ - most likely time/cost needed to complete the activity,
	\item $v\_max$ - maximum time/cost needed to complete the activity, 
	\item $P(V)$ - occurrence/probability
	\item In Monte Carlo function we work with Cumulative Distribution Function (CFD) which is the probability of the task completing by time T
	\item $P(V)$ is the Sum of all probabilities between $v\_min$ and $v$
	\item $P(V)_{CFD}$ should lie between 0 to 1.
\end{itemize}
     
Also, for methodology and simulation, refer to sections \ref{4.5.1} and \ref{4.5.2} in chapter 4. 

\subsection{Run simulation for variable - Time}
Input data : Time in days 

\begin{center}
	\begin{tabular}{ |c|c|c|c|c| c|} 
		\hline
		N&Activity & Minimum & Most Likely & Maximum\\
		\hline
		&1 & &  & \\ 
		\hline
		&2 & & &\\ 
		\hline
		&3& & &\\ 
		\hline
	\end{tabular}\label{table1}	
\end{center}

Steps for simulation in Excel for N = 10000 iterations:

\begin{enumerate}
	\item Create a series of activities within the Excel workbook using Table 1 defining the time as $v\_min$, $v\_ml$ and $v\_max$ needed for the completion of the entire project.
	
	\item Considering activity 1, Rows 2 to 5 in columns A and B shows the minimum, most likely and maximum completion variables and the same rows in Column C lists the probabilities for each variable.
	
For $v_ {min} $ the probability is 0 and for $v_ {max} $ is 1. 

The probability at $P (v_ {ml}) $ can be calculated by using equation \ref{eqn8} wherein $v = v_ {max} $ reduces the probability to 

\begin{equation}
p(v)_{ml}  = \frac{(v_{ml} - v_{min})}{  (v_{max} - v_{min}) } 
\label{eqn5.1}
\end{equation}

\item From Row 6 to 10005 in column A are simulated probabilities form activity 1. They are obtained by using MS-Excel RAND() function, which generates uniformly distributed random numbers between 0 and 1. This gives us a list of probabilities corresponding to 10000 independent iterations for an activity.

\item The 10000 probabilities are to be translated into completion variable for the activity. It is done by using equations \ref{eqn10} and \ref{eqn11} depending on whether the simulated probability is less than or greater than $P(v)_ {ml}$, which is in cell C3 and given by equation \ref{eqn12}. This conditional function is done in Excel by using IF () function.

\item Following all activities are simulated in a similar manner. Now let's combine them: Project had 41 broad activities.Tasks were both in series and parallel as shown in the Gantt chart. After simulating time for all the tasks individually, for all the tasks happening in parallel, only the maximum time is considered for the total time calculation. Critical Path is calculated as shown in table \ref{table2} below.

\begin{center}
	\begin{tabular}{ |c|c|c|c|c| c|} 

		\hline
		N&Activity & Minimum & Most Likely & Maximum\\
		\hline
		&1 & &  & \\ 
		\hline
		&2 & & &\\ 
		\hline
		&3& & &\\ 
		\hline
	\end{tabular}\label{table2}	
\end{center}

\item For activities in the series and the maximum duration of the parallel activities are added to get overall completion variables, which is shown in Rows 6 to 10005 of Columns from (H+K+Q+Z+AC+AL+AO+CB+CH+CK+DL+DR+DU+DX)
Finally, the overall project completion variable for each iteration is the sum of columns (H+K+Q+Z+AC+AL+AO+CB+CH+CK+DL+DR+DU+DX) and is shown in Column EF.

\item The minimum, average, median and maximum values of the completion times both in days and weeks are calculated in the entire iterations from Rows 6 to 10005 in column EH.

\item Sheets 2 and 3 shows the plots of the probability and cumulative probability distributions for complete project completion time and the results will be explained in Chapter 6.

\end{enumerate}

Similar simulation is run for the forecasted estimates for the cost for the project. The cost and schedule activities are calculated separately.

Table 3: Simulations for the cost.

\begin{center}
	\begin{tabular}{ |c|c|c|c|c| c|} 
		\hline
		N&Activity & Minimum & Most Likely & Maximum\\
		\hline
		&1 & &  & \\ 
		\hline
		&2 & & &\\ 
		\hline
		&3& & &\\ 
		\hline
	\end{tabular}\label{table3}	
\end{center}

\subsection{Run simulation for variable - Cost}

\textbf{Input data:} Cost in kCHF
\begin{center}
	\begin{tabular}{ |c|c|c|c|c|c|c|} 
		\hline
		Work Units&CB & IPP & 30/09/2019$\_$min & Moneyenne&Max\\
		\hline
		&1 & &  & &\\ 
		\hline
		&2 & & &&\\ 
		\hline
		&3& & &&\\ 
		\hline
	\end{tabular}\label{table4}	
\end{center}


Steps for simulation in Excel for N = 10000 iterations:

\begin{enumerate}
	\item Create a series of activities within the Excel workbook using Table 1 defining the time as $v\_min$, $v\_ml$ and $v\_max$ needed for the completion of the entire project.
	
	\item Considering activity 1, Rows 2 to 5 in columns A and B shows the minimum, most likely and maximum completion variables and the same rows in Column C lists the probabilities for each variable.
	
	For $v_ {min} $ the probability is 0 and for $v_ {max} $ is 1. 
	
	The probability at $P (v_ {ml}) $ can be calculated by using equation \ref{eqn8} wherein $v = v_ {max} $ reduces the probability to 
	
	\begin{equation}
	p(v)_{ml}  = \frac{(v_{ml} - v_{min})}{  (v_{max} - v_{min}) } 
	\label{eqn5.1}
	\end{equation}
	
	\item From Row 6 to 10005 in column A are simulated probabilities form activity 1. They are obtained by using MS-Excel RAND() function, which generates uniformly distributed random numbers between 0 and 1. This gives us a list of probabilities corresponding to 10000 independent iterations for an activity.
	
	\item The 10000 probabilities are to be translated into completion variable for the activity. It is done by using equations \ref{eqn10} and \ref{eqn11} depending on whether the simulated probability is less than or greater than $P(v)_ {ml}$, which is in cell C3 and given by equation \ref{eqn12}. This conditional function is done in Excel by using IF () function.
	
	\item Following all activities are simulated in a similar manner. Now let's combine them: Different work packages have several activates and all are simulated together for 10000 iterations in Row 6 to Row 10005.
	
	\item For activities in the series and the maximum duration of the parallel activities are added to get overall completion variables, which is shown in Rows 6 to 10005 of Columns from (AJ+CY+DA+DD+DG+DJ+DM+DP+DX+DZ).
	
	\item Finally, the overall project completion variable for each iteration is the sum of columns (AJ+CY+DA+DD+DG+DJ+DM+DP+DX+DZ) and is shown in Column EC.
	
	\item The minimum, average, median and maximum values of the completion times both in CHF and kCHF are calculated in the entire iterations from Rows 6 to 10005 in column ED.
	
	
	\item Sheets 2 and 3 shows the plots of the probability and cumulative probability distributions for complete project completion cost and the results will be explained in Chapter 6.
	
	
	Table 4 : Simulations with cost 
	
	
\end{enumerate}

After running the Monte-Carlo simulations, using the range from step 8, histogram is plotted for both time and cost with a defined bin width and then subsequently probability and cumulative probability distributions for complete project completion time and cost are evaluated with forecasted estimates most likely and then uncertainty is explained in an explicit manner.
