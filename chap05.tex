\let\textcircled=\pgftextcircled
\chapter{Case Study of Nuclear Infrastructure}
\label{chap:chap5}
The Isotope mass Separator On-Line facility (ISOLDE)\cite{Garcia_Ruiz_2020} at CERN is a "unique source of low-energy beams of radioactive nuclides, those with too many or too few neutrons to be stable". The facility fulfills in fact the old alchemical dream of changing one element into another. It enables  a wide ranging study of atomic nuclei, including the most exotic species. 

Owing to the volatility and pyrophoric properties of the radioactive particles, it is required to have dedicated laboratories to ensure safe handling of the radioactive particles. It is proposed to have an extension to existing "Class A laboratory" \cite{iaea} in order to provide safe space for handling and production of radioactive targets by building a dedicated space for carrying out the activities.

This decision for construction of Nano-Lab was taken by keeping in mind the benefits that can be availed from the already available infrastructure of the existing building.  A nuclear infrastructure project, the Nano-Lab, is studied during its design phase and currently in its execution phase. It is an extension of the existing building and will be comprised of a radioactive material storage area and two laboratories for manipulating uranium nano particles. The Nano-Lab project planning is the most critical step of the project management life cycle owing to the complex scientific environment at CERN and the potential risks associated to it. 

\section{Description of the project}

\textbf{Nano-Lab :}  A Nuclear infrastructure with a $22m * 7m$ building and a floor area of approximately 140$m^{2}$ and is fully reinforced concrete, partially composed of high density concrete walls ($> 3.9 tons/m^{3}$). It is an extension of existing building and the upcoming building will be composed of three sections:

\begin{enumerate}
	\item \textbf{Production Laboratory:} A dedicated laboratory for the production of radioactive nano-particles used as targets for ISOLDE facility.
	
	\item \textbf{Pump stand Laboratory:} is used for manipulation and calibration of the radioactive nano-particles.
	
	\item \textbf{Buffer Area:} A temporary storage area for radioactive waste and will be built with concrete possessing reinforced shielding properties. It will be used as an immediate repository for radioactive waste and also as a storage place for accommodating contaminated equipment from the ISOLDE experiment.
\end{enumerate}

\begin{figure}
	\centering
	\includegraphics[height=0.53\textheight]{fig05/img1chp5.png}
	\mycaption[Proposed Layout of the Nano-Lab]{Proposed Layout of the Nano-Lab}
\end{figure}

In the design phase of the project areas, the approximate areas calculated for the Nano-Lab layout are: 

\begin{enumerate}
	\item Buffer area: $49m^{2}$
	\item Pump stand Laboratory: $23m^{2}$
	\item Production Laboratory: $44m^{2}$
\end{enumerate}

\section {Forecasted Planning and Scheduling Phase} 

Forecasting and planning phase is carried out by the project manager assigned by the project management team. The project manager assigned to the Nano-Lab project is responsible for initiating, planning, executing, controlling and closing projects. This thesis aims to analyze the classical planning tools employed for this project, review their limitations and drawbacks, and study and implement alternative advanced planning methodologies. 

Project management is the application of knowledge, skills, tools, and techniques to the project activities in order to meet requirements. The focus of this thesis is on initiating and planning methodologies used for Nano-Lab project at CERN.

Project management processes \cite{PMBOK2013} fall into five groups:
\begin{itemize}
	\item  \textbf{Initiating}
	\item  \textbf{Planning}
	\item  Executing
	\item  Monitoring and Controlling
	\item  Closing
	
\end{itemize}

Project management knowledge \cite{PMBOK2013} draws on ten areas:
\begin{itemize}
	\item Scope
	\item \textbf{Time}
	\item \textbf{Cost}
	\item  Quality
	\item Human resources
	\item Communications
	\item Risk management
	\item Stakeholder management
	\item Integration
\end{itemize}

A Work Breakdown Structure (WBS) in project management and systems engineering is a deliverable-oriented breakdown of a project into smaller components. In the Nano-Lab project, the key project deliverables were divided into ten work packages. The WBS for the Nano-Lab displays the work packages involved and then the respective activities are defined accordingly.

A brief outline of work packages \cite{EN-Groups}, \cite{SMB:CERN} are:

\begin{enumerate}
	\item \textbf{Work Package$\_$1$\_$SMB:} Site Management and Building department (SMB) carry out the civil engineering activities, strategy, design and studies and further also carry out on-site execution works and integrate it with services provided by other work packages. The major activities they carry out are: 
	\begin{itemize}
		\item Conduct studies - Preliminary geo-technical study, design and technical studies. 
		
		\item Procurement activities.
		  
		\item Preliminary works on-site.
		
		\item Execute activities like earthwork, concrete and finishing's.
		
		\item Non-IT works in the Nano-Lab site. 
		
		\item Support safety coordinator activities.
	\end{itemize}
	
	\begin{figure}
		\centering
		\includegraphics[height=0.4\textheight]{fig05/img2chp5.png}
		\mycaption[Organizational Structure of the Nano-Lab Project]{Organizational Structure of the Nano-Lab Project}
	\end{figure}

	\item \textbf{Work Package$\_$2$\_$CV:} Cooling and Ventilation work packages is responsible for the design, construction, commissioning, maintenance and operations of the technical installations in Nano-Lab. The major activities they will carry out are: 
	\begin{itemize}
		\item Design documentation as-built and marking.
		
		\item Modifications in CV system in existing building. 
		
		\item Ventilation supply and extraction.
		
		\item Execute activities such as earthwork and concrete, finishing’s.
		
		\item Electrical works, instrumentation and control.
		
		\item Compression air works. 
		
		\item Pump stand, water cooling, raw water and leak detection. 
		
		\item Test and commissioning. 
		
		\item Process extraction and fans consolidation.
	\end{itemize}
	
	\item \textbf{Work Package$\_$3$\_$Process Equipment:} will be the users of the Nano-Lab who carry out the activities for particles production and carry out responsibilities of target stations for"'Class A" type laboratories. The major activities they will carry out are:

	\begin{itemize}
		\item Moving Pump stands.
		\item Delivery and installation of equipment in production laboratory. 
		\item Conduct equipment testing.
		
	\end{itemize}

	\item \textbf{Work Package$\_$4$\_$RP:} Radiation Protection work package carries out activities and services aimed at personnel protection, waste processing, radiation and stray monitoring and also establishes the rules and procedure with respect to the radiation protection. The major activities of this work package are:
	\begin{itemize}
		\item Supply of monitors for RP.
		\item Installation of Monitors.
		\item Conducting Tests.

	\end{itemize}
	
	\item \textbf{Work Package$\_$5$\_$EL:} Electrical group provides all electrical supplies, routings and installation, commissioning, and integration activities for electrical services. The major activities are:
	\begin{itemize}
		\item Installation support.
		\item Cable pulling’s for pump stand and fire detection.
		\item Installation of luminaires.
		\item Chute and catch chute installations.
		\item fire cable tests.
		\item Excavation of soil for Earthing services.	
	\end{itemize}
	
	\item \textbf{Work Package$\_$6$\_$IT:} Information Technology group provides all IT related services including supplies, installation and integration support to other work packages. The major activities they will carry out are: 
	
	\begin{itemize}
		\item Moving GSM cable.
		\item IT cable pulling.
		\item Works related to ducts and plug boxes.
		\item Conduct Tests.

	\end{itemize}
	
	\item \textbf{Work Package$\_$7$\_$FD:} Fire Detection provides the services for fire detection points, routings and procedures and safety aspects for the Nano lab protection for both personnel and equipment. The major activities they will carry out are:
	
	\begin{itemize}
		\item Installation of equipment and sniffers.
		\item Conduct Tests.
	\end{itemize}
	
	\item \textbf{Work Package$\_$8$\_$GAS:} Gas work package handles associated technical services for projects by providing gas supply and installation services. The major activities they will carry out are:
	
	\begin{itemize}
		\item Installation of gas pipelines for Argon. 
		\item  External work + tests.
	\end{itemize}
	
	\item \textbf{Work Package$\_$9$\_$AC:} Access Control package provides the access control services for the Nano-Lab to assist the users to ensure access to work in a controlled radioactive environment. The major activities they will carry out are: 
	
	\begin{itemize}
		\item Cable pull to door entry.
		\item Installation equipment and door connection.
		\item Conduct Tests.
	\end{itemize}
	
	\item \textbf{Work Package$\_$10$\_$PPM:} Project management team provides coordination, integration and monitoring activities apart from technical and administration support.
	
	
\end{enumerate}


\section{KPI for the project - Time and Cost}

Using WBS, the project planning is done for both time and cost KPIs. Historical data, technical and professional expertise are the foundations for forecasts related to time and costs. Project is divided into activities which are both in series and parallel. Planning is depicted via Gantt chart, a classical traditional project management tool for project planning \& scheduling. The Duration of the project will be Oct 2019 -June 2021 approximately 615 days. However, uncertainty comes into play, so rigorous governance of processes and control over data is needed to enhance reliability. Use of techniques like Monte-Carlo which is a powerful statistical analysis tool is used to understand the impact of risk and uncertainty in prediction and forecasting models.


\begin{figure}
	\centering
	\includegraphics[height=0.64\textheight]{fig05/img3chp5.png}
	\mycaption[Gantt Chart Schedule]{Gantt Chart Schedule}
\end{figure}



\subsection{Use of Monte-Carlo Simulation Model}\label{section5.3.1}

Use PERT like approach wherein estimates for an activity are based on 3 points in variables like time and cost for Nano-Lab project. Monte- Carlo simulation provides a principled and intuitive way to obtain probabilistic estimates at the level of an entire project based on estimates of the individual activities that comprise it. This three-point estimate will be the input for the Monte-Carlo simulation model. 

\begin{center}
	\begin{tabular}{ |c|c|c|c| c|} 
		\hline
		Activity & Minimum & Most Likely & Maximum\\
		\hline
		1 & &  & \\ 
			\hline
		2 & & &\\ 
			\hline
		3& & &\\ 
		\hline
	\end{tabular}
\end{center}

\begin{figure}
	\centering
	\includegraphics[height=0.12\textheight]{fig05/img4chp5.png}
	\mycaption[Monte Carlo Simulation Model]{Monte Carlo Simulation Model}
\end{figure}

\subsection{Input data for the model} \label{5.3.2}
	
The inputs are:
\begin{itemize}
	\item Variables (v): Time in days \& Cost in points scaled to 100	 for minimum, most likely and maximum points.
	\item $v\_min$ - minimum time/cost needed to complete the activity,
	\item $v\_ml$ - most likely time/cost needed to complete the activity,
	\item $v\_max$ - maximum time/cost needed to complete the activity, 
	\item $P(V)$ - occurrence/probability
	\item In Monte Carlo function we work with Cumulative Distribution Function (CFD) which is the probability of the task completing by time T
	\item $P(V)$ is the Sum of all probabilities between $v\_min$ and $v$
	\item $P(V)_{CFD}$ should lie between 0 to 1.
\end{itemize}
     
Also, for methodology and simulation, refer to sections \ref{4.5.1} and \ref{4.5.2} in chapter 4. 
\clearpage


\subsection{Run Simulation for Variable - Time}
Using input data from table \ref{tab:caption5}, the simulation was iterated for N = 10000 iterations and steps for undertaking the simulation in MS-Excel are as follows:

\begin{enumerate}
	\item Create a series of activities within the Excel workbook \cite{Github} using the table in MS-Excel defining the time as $v\_min$, $v\_ml$ and $v\_max$ needed for the completion of entire project.
	
	\item Considering activity 1, Rows 2 to 5 in columns A and B shows the minimum, most likely and maximum completion variables and the same rows in Column C lists the probabilities for each variable, with time being considered as the variable in this case. For $v_ {min} $ the probability is 0 and for $v_ {max} $ is 1. 
	The probability at $P (v_ {ml}) $ can be calculated by using equation \ref{eqn8} wherein $v = v_ {max} $ reduces the probability to 
	
	\begin{equation}
	p(v)_{ml}  = \frac{(v_{ml} - v_{min})}{  (v_{max} - v_{min}) } 
	\label{eqn5.1}
	\end{equation}
	
	\item From Row 6 to 10005 in column A are simulated probabilities form activity 1. They are obtained by using MS-Excel RAND() function, which generates uniformly distributed random numbers between 0 and 1. This gives us a list of probabilities corresponding to 10000 independent iterations for an activity.
	
	\item The 10000 probabilities are to be translated into completion times for the activity. It is done by using equations \ref{eqn10} and \ref{eqn11} depending on whether the simulated probability is less than or greater than $P(v)_ {ml}$, which is in cell C3 and given by equation \ref{eqn12}. This conditional function is done in Excel by using IF () function.
	
	\item Following all activities are simulated in a similar manner. Now let's combine them: Project has 41 broad activities with tasks being both in series and parallel as shown in the Gantt chart. After simulating time for all the tasks individually and also for all the tasks happening in parallel, only the maximum time is considered for the calculating total time. Network diagram is shown in image \ref{im7ch5} below, considering the overlaps in table \ref{tab:caption4} due to parallel activities where maximum duration is considered. 
	
	\begin{figure}
		\centering
		\includegraphics[height=0.32\textheight]{fig05/img7chp5.png}
		\mycaption[Network Diagram]{Network Diagram}
		\label{im7ch5}
	\end{figure}
	

\item For activities in the series and the maximum duration of the parallel activities are added to get overall completion variables, which is shown in Rows 6 to 10005 of Columns from (H+K+Q+Z+AC+AL+AO+CB+CH+CK+DL+DR+DU+DX). 

Finally, the overall project completion variable for each iteration is the sum of columns (H+K+Q+Z+AC+AL+AO+CB+CH+CK+DL+DR+DU+DX) and is shown in Column EF.

\item The minimum, average, median and maximum values of the completion times both in days and weeks are calculated in the entire iterations from Rows 6 to 10005 in column EH.

\item Sheets 2 and 3 in \cite{Github} shows the plots of probability and cumulative probability distributions for project completion time and the results are explained in Chapter 6.

\end{enumerate}

\begin{table}[ht]	
	\begin{center}
		\begin{tabular}{ |c|c| } 
			\hline
			List of Overlap Activities& Maximum Time (days) \\ 
			\hline
			5,6&  45\\ 
			\hline
			5,6,11&45   \\ 
			\hline
			5,6,12,15,16,19,29&45   \\ 
			\hline
			5,6,12,16,19&45   \\ 
			\hline
			6,17,18,22&45   \\ 
			\hline
			6,22,24&45   \\ 
			\hline
			25,34,35&15   \\ 
			\hline
			35,36,41&15   \\ 
			\hline
			26,27e,36,38&15   \\ 
			\hline
			7,28,38& 10   \\ 
			\hline
			39,40&10   \\ 
			\hline
			8,13,14,20,21,30&15   \\ 
			\hline
			13,14,31,32&15   \\ 
			\hline
			9,13,14,32&20   \\ 
			\hline
			9,33&20 \\ 
			\hline
			9,23&20 \\ 
			\hline
		\end{tabular}
	\end{center}
	\caption[List of Overlap Activities]{List of Overlap Activities}
	\label{tab:caption4}
\end{table}%	
\clearpage
\begin{table}[ht]
\begin{center}
	\begin{tabular}{ |c|c|c|c|c| c|} 
		\hline
		%\multicolumn{5}{|c|}{Forecasted Estimates of Time for all Work Package Activities} \\
		\hline
		N&Activity & Minimum & Most Likely & Maximum\\
		\hline
		1&SMB Strategy &10 &  15& 20\\ 
		\hline
		2&SMB Procurement Phase &5 &15 &20\\ 
		\hline
		3&SMB Phase 1 works&132 &176 &220\\ 
		\hline
		4&HVAC Pipes support and equipment &25&35&45\\
		\hline
		5&HVAC electrical cable&25&30&40\\
		\hline
		6&HVAC existing connections&40&45&50\\
		\hline
		7&HVAC new process equipment connections&1&3&5\\
		\hline
		8&HVAC Ventilation grills&1&3&5\\
		\hline
		9&Commissioning&15&20&30\\
		\hline
		10&Fire Alarm Tests&1&1&5\\
		\hline
		11&EL Installation Support&5&10&20\\
		\hline
		12&EL Electrical Cable Installation &5&10&20\\
		\hline
		13&EL Lighting&10&15&20\\
		\hline
		14&EL Cables on Walls&10&15&20\\
		\hline
		15&EL/FC Pump Stand Cables&1&5&10\\
		\hline
		16&EL/FC Fire Electrical Cable&5&10&20\\
		\hline
		17&Fire Electrical Cable&1&5&5\\
		\hline
		18&IT GSM Cable&5&5&10\\
		\hline
		19&IT Electrical Cables&5&10&15\\
		\hline
		20&IT Electrical Work on Walls&1&3&5\\
		\hline
		21&IT Electrical Cable Tests &1&1&3\\
		\hline
		22&Fire Detection Equipment&10&15&20\\
		\hline
		23&Fire Detection Tests&1&1&5\\
		\hline
		24&Gas Installation&5&10&15\\
		\hline
		25&Gas External Works&3&5&10\\
		\hline
		26&Moving Pump Stands&3&5&10\\
		\hline
		27a&Glove Box Spec Tech &20&34&40\\
		\hline
		27b&Glove Box Procurement Phase&45&45&60\\
		\hline
		27c&Glove Box Design&30&30&50\\
		\hline
		27d&Glove Box Making&70&70&100\\
		\hline
		27e&Process Equipment Installation&1&3&5\\
		\hline
		28&Test Equipment&1&1&3\\
		\hline
		29&AC Electrical Cables&5&5&15\\
		\hline
		30&AC Door Connections&1&3&5\\
		\hline
		31&AC Tests&1&2&5\\
		\hline
		32&RP Monitor Installation&10&12&15\\
		\hline
		33&RP Tests&10&12&15\\
		\hline
		34&Demolish Existing \& Temporary Doors&5&10&20\\
		\hline
		35&Resin&10&15&25\\
		\hline
		36&Painting&10&15&25\\
		\hline
		37&Resin on Equipment&1&3&5\\
		\hline
		38&Light, Walls and Doors&5&10&20\\
		\hline
		39&False Ceiling&5&10&15\\
		\hline
		40&Last Doors&3&5&10\\
		\hline
		41&Fire Stop Fillers&3&5&10\\
		\hline
	\end{tabular}
\end{center}

	\caption[Forecasted Estimates of Time for all Work Package Activities]{Forecasted Estimates of Time for all Work Package Activities}
\label{tab:caption5}
\end{table}%


The cost and schedule activities are calculated separately. Similarly, simulation is run for the forecasted estimates for the cost of the project. Input data used in terms of cost in the Monte-Carlo simulation model is carried out analogously as in section \ref{5.3.2}.  


\subsection{Run simulation for Variable - Cost}

\textbf{Input data:}  Due to confidentiality reasons the values for cost are assumed on the scale of 100 points for the purpose of calculation and are hence values are fictitious in nature.

\begin{table}[ht]
	
\begin{center}
	\begin{tabular}{ |c|c|c|c|c|} 
		\hline
		%\multicolumn{4}{|c|}{Forecasted Cost Estimates for all Work Package Activities} \\
		\hline
		
		Work Units 													& Minimum & Most  Likely&Maximum\\
		\hline
		\colorbox{Melon}{Total Civil Engineering}	&26.6 &28.2&29.8\\ 
		\hline
		\colorbox{Melon}{Studies}							&5.5&5.5&5.5\\ 
		\hline
		\colorbox{Melon}{Preliminary Works}			&6&6.4&6.8\\ 
		\hline
		\colorbox{Melon}{Earth Works and Concrete}	&5.2 &5.7&6.3\\
		\hline
		\colorbox{Melon}{Finishing}							&6.3&6.7&7\\ 
		\hline
		\colorbox{Melon}{Services for IT}				&0.8&1.1&1.4\\
		\hline
		\colorbox{Melon}{Draftsman/Work Supervisor}		&2.8&2.8&2.8\\
		\hline
		\colorbox{Gray}{HVAC}								&31.2 &35.1& 38.8\\
		\hline
		\colorbox{Gray}{Replacement of Process Extractor}	&1.2& 1.3&1.5\\
		\hline
		\colorbox{Gray}{Duct Work and Diffusers}		&2.2&2.6&3.2\\
		\hline
		\colorbox{Gray}{Pipe Work Process}		&2& 2.3&2.6\\
		\hline
		\colorbox{Gray}{Cooling Pump Stand}	&1.6&1.7&1.8\\
		\hline
		\colorbox{Gray}{Compressed Air}			&0.5&0.5&0.6\\
		\hline
		\colorbox{Gray}{Instrumentation}		&2.4&2.5&2.6\\
		\hline
		\colorbox{Gray}{Accessories}			&1.1&1.2&1.2\\
		\hline
		\colorbox{Gray}{Filters}					&3.4&3.8&4.2\\
		\hline
		\colorbox{Gray}{Modification existing Power Cubicle}	&0.5  & 		0.5			&0.6\\
		\hline
		\colorbox{Gray}{Control Cubicle}		&1.3   							& 		1.4			&1.5\\
		\hline
		\colorbox{Gray}{Electric Cabling- Earthing \& Carriers}	&1.8 			& 	2.2				&2.6\\
		\hline
		\colorbox{Gray}{Design. Documentation as built and Marking}	&2.2  			& 		2.5			&2.8\\
		\hline
		\colorbox{Gray}{FAT, Testing and Commissioning }		&1.8   							& 	2.3				&2.6\\
		\hline
		\colorbox{Gray}{Transport, Manual Handling }	&1.0   & 1.2	&1.5\\
		\hline
		\colorbox{Gray}{Others}			&0.6   		& 	0.6			&0.7\\
		\hline
		\colorbox{Gray}{Scaffolding}		&0.3   							& 	0.4				& 0.5\\
		\hline
		\colorbox{Gray}{FSU Control Software}		&1.3   							& 		1.5			&1.5\\
		\hline
		\colorbox{Gray}{FSU Supervision During Work}	&1.8   		& 		1.9			&2.0\\
		\hline
		\colorbox{Gray}{FSU Draftsman}		&1.6   			& 	1.6				&1.7\\
		\hline
		\colorbox{Gray}{FSU Documentation and Inventory}	&0.8  							& 			1.0		&1.1\\
		\hline
		\colorbox{Gray}{BAG Oxidation}										&0   							& 	0				&0\\
		\hline
		\colorbox{Gray}{Modification Existing Rooms}	&2.0 							& 		2.1			&2.2\\
		\hline
		\colorbox{Melon}{Electricity}		&1.8   							& 	2.2				&2.6\\
		\hline
		\colorbox{Gray}{IT}		&1.3   							& 	1.5				&1.7\\
		\hline
		\colorbox{Melon}{RP Monitoring}		&6  							& 	6.2				&6.4\\
		\hline
		\colorbox{Gray}{Fire Detection}			&2.8  							& 	3.5				&4.2\\
		\hline
		\colorbox{Melon}{Gas }		&3.5					& 			3.9		&4.2\\
		\hline
		\colorbox{Gray}{Access Control}			&1.6						& 	1.7					&1.8   \\
		\hline
		\colorbox{Melon}{Laboratory Equipment}			&4   							& 	4.9				&5.8\\
		\hline
		\colorbox{Gray}{Resource Project Management}			&4.5							& 	4.5			&4.5\\
		\hline
		Total Project 	Cost								&83.3							& 91.8				&99.9\\
		\hline
	\end{tabular}
\end{center}
\caption[Forecasted Cost Estimates for Work Package Activities]{Forecasted Cost Estimates for Work Package Activities}
\label{tab:caption2}
\end{table}%

Steps for simulation in Excel for N = 10000 iterations using data as input from table \ref{tab:caption2}:

\begin{enumerate}
	\item Create a series of activities within the Excel workbook \cite{Github} using Table 1 defining the time as $v\_min$, $v\_ml$ and $v\_max$ needed for the completion of the entire project.
	
	\item Considering activity 1, Rows 2 to 5 in columns A and B shows the minimum, most likely and maximum completion variables and the same rows in Column C lists the probabilities for each variable.
	
	For $v_ {min} $ the probability is 0 and for $v_ {max} $ is 1. 
	
	The probability at $P (v_ {ml}) $ can be calculated by using equation \ref{eqn8} wherein $v = v_ {max} $ reduces the probability to 
	
	\begin{equation}
	p(v)_{ml}  = \frac{(v_{ml} - v_{min})}{  (v_{max} - v_{min}) } 
	\label{eqn5.1}
	\end{equation}
	
	\item From Row 6 to 10005 in column A are simulated probabilities form activity 1. They are obtained by using MS-Excel RAND() function, which generates uniformly distributed random numbers between 0 and 1. This gives us a list of probabilities corresponding to 10000 independent iterations for an activity.
	
	\item The 10000 probabilities are to be translated into completion variable for the activity. It is done by using equations \ref{eqn10} and \ref{eqn11} depending on whether the simulated probability is less than or greater than $P(v)_ {ml}$, which is in cell C3 and given by equation \ref{eqn12}. This conditional function is done in Excel by using IF () function.
	
	\item Following all activities are simulated in a similar manner. Now let's combine them: Different work packages have several activates and all are simulated together for 10000 iterations in Row 6 to Row 10005.
	
	\item For activities in the series and the maximum duration of the parallel activities are added to get overall completion variables, which is shown in Rows 6 to 10005 of Columns from (AJ+CY+DA+DD+DG+DJ+DM+DP+DX+DZ).
	
	\item Finally, the overall project completion variable for each iteration is the sum of columns (AJ+CY+DA+DD+DG+DJ+DM+DP+DX+DZ) and is shown in Column EC.
	
	\item The minimum, average, median and maximum values of the completion cost in points scaled to 100 is calculated in all the iterations from Rows 6 to 10005 in column ED.
	
	\item Sheets 2 and 3 shows the plots of the probability and cumulative probability distributions for complete project completion cost and the results will be explained in Chapter 6.
	
	
\end{enumerate}

After running the Monte-Carlo simulations, using the range from step 8, histogram is plotted for both time and cost with a defined bin width and then subsequently probability and cumulative probability distributions for complete project completion time and cost are evaluated with forecasted estimates most likely and then uncertainty is explained in an explicit manner.
