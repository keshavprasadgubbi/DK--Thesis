\let\textcircled=\pgftextcircled
\chapter{Conclusions}
\label{chap:annex}
%\section{Conclusions}
Monte-Carlo analysis is a powerful tool that can give project controls professionals and project managers valuable information that can be used to improve project execution by identifying risks and the most important activities within a schedule that may not be on the critical path, ensuring a realistic execution plan, aligning the stakeholders involved in a project, and avoiding claims and potential liquidated damages in complex projects. Historical estimates and human judgement with technical and professional expertise and classical project planning tools are an important tool for project planning of complex and scientific projects.

It helps to understand the uncertainty which is deciphered from the region in the plot for CDF, where probability changes most rapidly with time. As I have presumed the assumptions made for 3-point estimates by the project manager or team are valid in case of complex scientific projects where not much of the relevant data is available from outside of the organization. Hence the exact results of the simulations are dependent on these human-made assumptions.

Furthermore, Monte-Carlo Technique simplified MS-Excel tools helps in optimization of the planning \& scheduling and provide results that are easy to analyses and understand (schedule for start-finish of the project). This in turn helps to validate the results with the manual forecasts for project based on human skills. Further as recommended, the sensitivity of the activities involved can also be analyzed through Monte-Carlo analysis by understanding what a schedule is sensitive to, it can allow changes to execution to improve the end date. It will contribute to development of realistic or achievable payment milestones within a schedule. As in recent challenging times of COVID-19, the projects are delayed due to unplanned suspension of activities which would have significantly affected the Projects schedule and end-dates. So Monte-Carlo as an advanced planning methodology can help to evaluate the present completion percentage and progress of the projects in terms of time or cost as during estimates we already consider a range and it makes it easy to update the project planning and schedule accordingly and monitor the work packages deadlines and payment milestones.

Monte-Carlo analysis will never replace good judgment and the human ability to make decisions based on experience but a combination of two is an incredible method to govern processes and control variables for project management.