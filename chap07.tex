\let\textcircled=\pgftextcircled
\chapter{Proposals and Recommendations}
\section{Proposals} 
As discussed in the chapter 1 and in chapter 6, one of the risks associated with usage of traditional classical project planning methodologies is the project delays and cost overruns. As a further step or an extension to the work carried out in this thesis, Monte-Carlo Simulation technique can be further utilized to analyze the risk associated with different aspects of the project. 

Uncertainty is always associated with estimates for the total time and the overall cost of the project as discussed as the main focus if this thesis. Currently, there are several different methodologies available to conduct risk analysis, the most common amongst them are Sensitivity Analysis using Monte-Carlo and Earned Value Management, EVM. Sensitivity Analysis is based on the dynamic aspects including risks associated to the project and EVM is primarily focused on the historical data. The two methodologies are explained in brief below. 

\subsection{Sensitivity Analysis}

It is the Quantitative risk assessment\cite{Wood2002RiskST}  which evaluates how changes in a specific model variable impacts the output of the model. For example in project management it allows to identify which tasks duration with uncertainty has most impact or a stronger correlation with the end date of the project.

As functions of time and cost activities contributes to the total risk, using Monte-Carlo, sensitivity analysis can be done through Tornado diagram which shows how sensitive is the project to the each decision variable as they change over allowed ranges, how they contribute towards final risk associated to total project completion  time and cost. It acts as powerful tool for decision makers as it not only focuses on the most important activities but also guide them to take necessary action based on the probabilities calculated through Monte-Carlo method. It is highly beneficial as it provides a dynamic analysis which is easily understood and efficient in explaining uncertainty and give results which are easily intercepted by project team in order to take necessary actions.

\begin{figure}
	\centering
	\includegraphics[height=0.75\textwidth]{fig07/fig1ch7.png}
	\mycaption[Tornado Diagram for Sensitivity Analysis]{Tornado Diagram for Sensitivity Analysis}
	\label{img14ch7}
\end{figure}



\subsection{Earned Value Management (EVM)}

EVM is the assessment of actual successful completion of the project and its translation into a metric called earned value, which provide project manager a greater insight into both project performance and potential risk areas.

The key to the early visibility and the resultant management opportunities is the existence of an integrated baseline. Such a baseline results from a four-step process. First the work of the project is comprehensively defined and subdivided into commonly understood tasks. These tasks are phased consistent with the performance sequence logic and the schedule objectives of the project. Resource estimates are then established for the performance of the tasks and are designated as the budgets or targets for the work; these are time phased consistent with management visibility needs of the project. Finally, this is integrated with the risk assessment/management process.


\subsubsection{How to Calculate Earned Value \cite{Sparrow}}

Earned value calculations require the following:

Planned Value (PV) = the budgeted amount through the current reporting period
Actual Cost (AC) = actual costs to date
Earned Value (EV) = total project budget multiplied by the % of project completion

With these readily available information, can do more calculations.

\textbf{Schedule Performance Index (SPI)}

 SPI = $\frac{EV}{PV}$

SPI measures progress achieved against progress planned. An SPI value <1.0 indicates less work was completed than was planned. SPI >1.0 indicates more work was completed than was planned.

\textbf{Cost Performance Index (CPI)} 

CPI =  $\frac{EV}{AC}$

CPI measures the value of work completed against the actual cost. A CPI value <1.0 indicates costs were higher than budgeted. CPI >1.0 indicates costs were less than budgeted.

For both SPI and CPI, >1 is good, and <1 is bad. 

\textbf{Estimated at Completion (EAC)} 

EAC = $\frac{(Total Project Budget)}{CPI}$

EAC is a forecast of how much the total project will cost.

\begin{figure}
	\centering
	\includegraphics[height=0.35\textheight]{fig07/fig2ch7.png}
	\mycaption[Probability distribution of completion times (N = 10000)]{Probability distribution of completion times (N = 10000)}
	\label{img15ch7}
\end{figure}


\subsection {Limitations} 

One of the main limitations of EVM in comparison to Monte-Carlo tools for Risk analysis, are EVM may fail to consider the dynamic changes happening.
Using EVM for project planning and monitoring tool, if there are projects delays or project run beyond the budget, it may lead to misleading figures for Project KPIs when it comes to forecasts and estimates. Another issue is that EVM is significantly based on the assumption that future performance of the different aspects of the project can be estimated on the basis of the past performance. The different performance indicators of EVM such as CPI, SPI etc. are optimized not only to predict the future [parameters, but also to make optimal decisions. Further, there is a greater possibility that the actual values of different variables in future may deviate from the estimated values as forecasts are based on historical performance.

\section{Recommendations}

Hence a combination of the Monte-carlo for project planning, risk analysis and project performance monitoring can be efficiently dine by using EVM.
It may prove to be an effective project management methodology to handle complex projects based strongly on historical estimates and ensure the project meets both schedule and budget deadlines and can adapt to the uncertainties.



